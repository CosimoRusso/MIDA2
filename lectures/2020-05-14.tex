\newlecture{Sergio Savaresi}{14/05/2020}

K.F. is a sophisticated way to build those T.F. from a white-box model.
We can estimate these functions directly from data.

We can adopt a black-box approach to estimate these T.F.:

\paragraph{Dataset for training}
\begin{align*}
    \left\{ u(1), u(2), \ldots, u(N) \right\} \\
    \left\{ y(1), y(2), \ldots, y(N) \right\} \\
    \left\{ x(1), x(2), \ldots, x(N) \right\}
\end{align*}

In the supervised training approach we need measurements of the state to be estimated (physical sensor for $x(t)$ only for design/training of the software-sensor).

We can use 4SID for direct non-parametric identification of $(u,y)\rightarrow x$ dynamics, or we can use classic parametric system identification approach:

\paragraph{Model selection} \phantom{lol}

\begin{figure}[H]
    \centering
    \begin{tikzpicture}[node distance=2cm,auto,>=latex']
        \node[int] (ux) {$s_{ux}(z, \theta)$};
        \node[int, below of=ux] (yx) {$s_{yx}(z, \theta)$};
        \node[left,left of=ux] (u) {$u(t)$};
        \node[left,left of=yx] (y) {$y(t)$};
        \node[sum,right of=ux,xshift=1cm,yshift=-1cm] (sum) {};
        \node[right, right of=sum] (x) {$\hat{x}(t)$};

        \draw[->] (u) -- (ux);
        \draw[->] (y) -- (yx);
        \draw[->] (ux) -| (sum);
        \draw[->] (yx) -| (sum);
        \draw[->] (sum) -- (x);
    \end{tikzpicture}
\end{figure}

\paragraph{Performance index}
\[
    J_N(\theta) = \frac{1}{N}\sum_{t=1}^N \left( x(t) - (S_{ux}(z, \theta) u(t) + S_{yx}(z,\theta)y(t)) \right)^2
\]

\paragraph{Optimization}
\[
    \hat{\theta}_N = \argmin_\theta J_N(\theta)
\]

We obtain $S_{ux}(z, \hat{\theta}_N)$ and $S_{yx}(z, \hat{\theta}_N)$, the black-box software sensor.

\section{Comparison between K.F. and B.B. software sensing}

\begin{center}
    \begin{tabular}{l|c|c}
        & \textbf{K.F.} & \textbf{B.B.} \\
        \hline
        Need of (WB) physical model of the system & \color{red} Yes & \color{green} No \\
        Need a training set & \color{green} No & \color{red} Yes \\
        Interpretability of the result & \color{green} Yes & \color{red} No \\
        Easy retuning for a similar (different) system & \color{green} Yes & \color{red} No \\
        Accuracy of the estimation & \color{green} Good & \color{green} Very Good \\
        Can be used also in case of un-measurable states & \color{green} Yes & \color{red} No \\
    \end{tabular}
\end{center}

\section{Non-linear Systems}

When the system is non-linear the problem becomes more complicated.
Let's take inspiration from E.K.F.

\begin{figure}[H]
    \centering
    \begin{tikzpicture}[node distance=2cm,auto,>=latex']
        \node[int,dashed border,align=center] at (2,4) (n1) {non-lin\\dyn sys};
        \node[int,double border,align=center] at (2,2) (n2) {non-lin\\function};
        \node[int,dashed border,align=center] at (2,0) (n3) {non-lin\\dyn sys};
        \node[sum] at (4,2) (sum) {};
        \node[left] at (0,4) (u) {input};
        \node[left] at (6,4) (y) {output};

        \draw[dashed, blue] (0,-1) rectangle (5,3);

        \draw[->] (u) -- (n1);
        \draw[->] (0.5,4) |- (n3);
        \draw[->] (n2) -- (n3);
        \draw[->] (sum) -- (n2);
        \draw[->] (n1) -| (sum) node[pos=0.9] {+};
        \draw[<-] (sum) |- (n3) node[pos=0.1] {-};
        \draw[->] (n1) -- (y);
        \draw[->,transform canvas={yshift=-0.3cm}] (n3) -- ++(4,0) node[right] {$\hat{x}(t)$};
    \end{tikzpicture}
\end{figure}

\begin{remark}
    In K.F. the E.K.F. extension uses the trick of a time-varying linear gain $K(t)$ but the obvious choice is a non-linear gain (static nonlinear function).
\end{remark}

The content of the box is:

\begin{figure}[H]
    \centering
    \begin{tikzpicture}[node distance=2cm,auto,>=latex']
        \node[int,double border,align=center] at (0,0) (n) {non-lin\\dyn T.I. sys};
        \draw[<-,transform canvas={yshift=0.3cm}] (n) -- ++(-2,0) node[left] {$u(t)$};
        \draw[<-,transform canvas={yshift=-0.3cm}] (n) -- ++(-2,0) node[left] {$y(t)$};
        \draw[->] (n) -- ++(2,0) node[right] {$\hat{x}(t)$};
    \end{tikzpicture}
\end{figure}

The problem is again the B.B. identification of a non-linear dynamic system, starting from a measured training dataset.

\paragraph{Architecture \#1} Use a recurrent neural network

\begin{figure}[H]
    \centering
    \begin{tikzpicture}[node distance=2cm,auto,>=latex']
        \node[int,dashed border,align=center] at (0,0) (n) {recurrent\\neural net};
        \draw[<-,transform canvas={yshift=0.3cm}] (n) -- ++(-2,0) node[left] {$u(t)$};
        \draw[<-,transform canvas={yshift=-0.3cm}] (n) -- ++(-2,0) node[left] {$y(t)$};
        \draw[->] (n) -- ++(2,0) node[right] {$\hat{x}(t)$};
    \end{tikzpicture}
\end{figure}


\begin{figure}[H]
    \centering
    \begin{minipage}[t]{0.48\textwidth}
        \centering
        \begin{tikzpicture}[node distance=2.5cm,auto,>=latex']
            \node[int] at (0,4) (a1) {$a_1$};
            \node[int] at (0,3) (a2) {$a_2$};
            \node at (0,2) {$\vdots$};
            \node[int] at (0,1) (ah) {$a_h$};
            \node[sum] at (2,2) (sum) {};
            \node[int,ellipse,align=center] at (3.5,2) (nlf) {N.L.\\fun};

            \draw[<-] (a1) -- ++(-1,0);
            \draw[<-] (a2) -- ++(-1,0);
            \draw[<-] (ah) -- ++(-1,0);

            \draw[->] (a1) -- (sum);
            \draw[->] (a2) -- (sum);
            \draw[->] (ah) -- (sum);
            \draw[<-] (sum) -- ++(0,0.5) node[above] {$b$};
            \draw[->] (sum) -- (nlf);
            \draw[->] (nlf) -- ++(1.5,0);
        \end{tikzpicture}
        \caption*{Static function of a neuron}
    \end{minipage}
    \begin{minipage}[t]{0.48\textwidth}
        \centering
        \begin{tikzpicture}[node distance=1.5cm,auto,>=latex']
            \node[int] at (0,4) (a1) {$a_1$};
            \node[int] at (0,3) (a2) {$a_2$};
            \node at (0,2) {$\vdots$};
            \node[int] at (0,1) (ah) {$a_h$};
            \node[sum] at (2,2) (sum) {};
            \node[int,ellipse,align=center] at (3.5,2) (nlf) {N.L.\\fun};
            \node[int, below of=nlf] (z) {$z^{-1}$};
            \node[int, below of=sum] (c) {$c$};

            \draw[<-] (a1) -- ++(-1,0);
            \draw[<-] (a2) -- ++(-1,0);
            \draw[<-] (ah) -- ++(-1,0);

            \draw[->] (a1) -- (sum);
            \draw[->] (a2) -- (sum);
            \draw[->] (ah) -- (sum);
            \draw[<-] (sum) -- ++(0,0.5) node[above] {$b$};
            \draw[->] (sum) -- (nlf);
            \draw[->] (nlf) -- ++(1.5,0);

            \draw[->] (4.5,2) |- (z);
            \draw[->] (z) -- (c);
            \draw[->] (c) -- (sum);
        \end{tikzpicture}
        \caption*{Upgrade to a dynamic neuron}
    \end{minipage}
\end{figure}

Most general approach but practically seldom used: major issues of stability and convergence of training.

\paragraph{Architecture \#2} Split the system into a static non-linear system and linear dynamics (F.I.R. architecture)

\begin{figure}[H]
    \centering
    \begin{tikzpicture}[node distance=2cm,auto,>=latex']
        \node[int] at (1.5,0) (yn) {$z^{-1}$};
        \node[int] at (1.5,1.5) (y2) {$z^{-1}$};
        \node[int] at (1.5,2.5) (y1) {$z^{-1}$};

        \node[int] at (1.5,4) (un) {$z^{-1}$};
        \node[int] at (1.5,5.5) (u2) {$z^{-1}$};
        \node[int] at (1.5,6.5) (u1) {$z^{-1}$};

        \node[left] at (0,6.5) (u) {$u(t)$};
        \node[left] at (0,3.2) (y) {$y(t)$};

        \node[int,minimum height=7cm,double border,align=center] at (5.25,3.25) (sys) {non-linear\\static\\parametric\\function\\(e.g. static\\neural\\network)};

        \draw[->] (u) -- (u1);
        \draw[->] (u1) -- (u2);
        \draw[->,dotted] (u2) -- (un);
        \draw[->] (u1) -- (u1-|sys.west) node[pos=0.5] {$u(t-1)$};
        \draw[->] (u2) -- (u2-|sys.west) node[pos=0.5] {$u(t-2)$};
        \draw[->] (un) -- (un-|sys.west) node[pos=0.5] {$u(t-n_u)$};

        \draw[->] (y) -- (y-|sys.west);
        \draw[->] (1.5,3.2) -- (y1);
        \draw[->] (y1) -- (y2);
        \draw[->,dotted] (y2) -- (yn);
        \draw[->] (y1) -- (y1-|sys.west) node[pos=0.5] {$y(t-1)$};
        \draw[->] (y2) -- (y2-|sys.west) node[pos=0.5] {$y(t-2)$};
        \draw[->] (yn) -- (yn-|sys.west) node[pos=0.5] {$y(t-n_y)$};

        \draw[->] (sys) -- ++(2,0) node[right] {$\hat{x}(t)$};

        \draw[decoration={brace}, decorate] (3.5,-0.7) node {} -- (0,-0.7);
        \node[align=center,below] at (1.75,-0.9) {linear dynamic\\system};

        \draw[decoration={brace}, decorate] (7,-0.7) node {} -- (3.6,-0.7);
        \node[align=center,below] at (5.25,-0.9) {non-linear static\\system to be\\estimated};
    \end{tikzpicture}
\end{figure}

\begin{remark}
    Notice that in principle $\hat{x}(t)$ can depend on $y(t)$ whereas we know $\hat{x}(t)$ can only depend on $u(t-1)$ and past values.

    In case of a MIMO system with
    \begin{align*}
        m \text{ inputs: } u(t) = \begin{bmatrix}
            u_1(t) \\
            \vdots \\
            u_m(t)
        \end{bmatrix} \quad p \text{ outputs: } y(t) = \begin{bmatrix}
            y_1(t) \\
            \vdots \\
            y_p(t)
        \end{bmatrix} \quad n \text{ states: } x(t) = \begin{bmatrix}
            x_1(t) \\
            \vdots \\
            x_n(t)
        \end{bmatrix}
    \end{align*}

    The estimation problem is the search of the optimal parameter vector $\theta$ for the function
    \[
        f(\cdot, \theta): \RR^{m\times n_u + p \times (n_y + 1)} \rightarrow \RR^n
    \]

    The estimation of this function $f(\cdot, \theta)$ is much simpler than the estimation of a recurrent neural network.
    Moreover the stability is guaranteed (the system is F.I.R.).
\end{remark}

\paragraph{Architecture \#3} Static non-linear function plus linear dynamics but with a I.I.R. scheme

\begin{figure}[H]
    \centering
    \begin{tikzpicture}[node distance=2cm,auto,>=latex']
        \node[int] at (1.5,0) (yn) {$z^{-1}$};
        \node[int] at (1.5,1.5) (y2) {$z^{-1}$};
        \node[int] at (1.5,2.5) (y1) {$z^{-1}$};

        \node[int] at (1.5,4) (un) {$z^{-1}$};
        \node[int] at (1.5,5.5) (u2) {$z^{-1}$};
        \node[int] at (1.5,6.5) (u1) {$z^{-1}$};

        \node[int] at (1.5,8) (xn) {$z^{-1}$};
        \node[int] at (1.5,9.5) (x2) {$z^{-1}$};
        \node[int] at (1.5,10.5) (x1) {$z^{-1}$};

        \node[left] at (0,10.5) (x) {$x(t)$};
        \node[left] at (0,6.5) (u) {$u(t)$};
        \node[left] at (0,3.2) (y) {$y(t)$};

        \node[int,minimum height=11cm,minimum width=1.5cm,double border,align=center] at (5.25,5.25) (sys) {$f(\cdot,\theta)$};

        \draw[->] (u) -- (u1);
        \draw[->] (u1) -- (u2);
        \draw[->,dotted] (u2) -- (un);
        \draw[->] (u1) -- (u1-|sys.west) node[pos=0.5] {$u(t-1)$};
        \draw[->] (u2) -- (u2-|sys.west) node[pos=0.5] {$u(t-2)$};
        \draw[->] (un) -- (un-|sys.west) node[pos=0.5] {$u(t-n_u)$};

        \draw[->] (y) -- (y-|sys.west);
        \draw[->] (1.5,3.2) -- (y1);
        \draw[->] (y1) -- (y2);
        \draw[->,dotted] (y2) -- (yn);
        \draw[->] (y1) -- (y1-|sys.west) node[pos=0.5] {$y(t-1)$};
        \draw[->] (y2) -- (y2-|sys.west) node[pos=0.5] {$y(t-2)$};
        \draw[->] (yn) -- (yn-|sys.west) node[pos=0.5] {$y(t-n_y)$};

        \draw[->] (x) -- (x1);
        \draw[->] (x1) -- (x2);
        \draw[->,dotted] (x2) -- (xn);
        \draw[->] (x1) -- (x1-|sys.west) node[pos=0.5] {$x(t-1)$};
        \draw[->] (x2) -- (x2-|sys.west) node[pos=0.5] {$x(t-2)$};
        \draw[->] (xn) -- (xn-|sys.west) node[pos=0.5] {$x(t-n_u)$};

        \draw[->] (sys) -- ++(2,0) node[right] {$\hat{x}(t)$};

        \draw[dashed, blue] (0.7,7.2) rectangle (4.2,11.5) node[black, right] {recursive I.I.R. architecture};
    \end{tikzpicture}
\end{figure}

\paragraph{Architecture \#4} Separation of system dynamics and a static non-linear system using regressors built from physical knowledge

\begin{figure}[H]
    \centering
    \begin{tikzpicture}[node distance=2cm,auto,>=latex']
        \node[int, dashed border, minimum width=1.5cm, minimum height=3cm] at (0,0) (sys) {};
        \node[int, dashed border, minimum height=3cm] at (4,0) (f) {$f(\cdot,\theta)$};

        \draw[<-,transform canvas={yshift=0.5cm}] (sys) -- ++(-2cm,0) node[left] {$u(t)$};
        \draw[<-,transform canvas={yshift=-0.5cm}] (sys) -- ++(-2cm,0) node[left] {$y(t)$};

        \draw[->,transform canvas={yshift=1.2cm}] (sys) -- (f) node[pos=0.5] {$r_1(t)$};
        \draw[->,transform canvas={yshift=0.6cm}] (sys) -- (f) node[pos=0.5] {$r_2(t)$};
        \draw[->,transform canvas={yshift=-1.2cm}] (sys) -- (f) node[pos=0.5] {$r_h(t)$};
        \node at (2,0) {$\vdots$};
        \draw[->] (f) -- ++(2cm,0) node[right] {$\hat{x}(t)$};
    \end{tikzpicture}
\end{figure}

The system (can be dynamic and non-linear) that builds the regressors (signals $r_1(t)$, $r_2(t)$, \ldots) from the physical sensors $u(t)$ and $y(t)$ using physical knowledge of the system.

The idea is to facilitate the job of $f(\cdot, \theta)$ by presenting at its input a smaller and more meaningful set of signals (regressors).

\begin{exercise}[Continue from the last one]
    Model (key equation) of the system:
    \[
        M\ddot{z} = -c(t)(\dot{z}-\dot{z}_d) - K(z-z_d)
    \]

    Measurable input $\ddot{z}$ with an accelerometer, $z-z_d$ measurable output with elongation sensor.
    We want to estimate $\dot{z}$.

    The change is $c(t)$ is a semi-active suspension, can be electronically changed (control variable).

    We can solve the problem with K.F. or we can make an experiment and collect training data:
    \begin{align*}
        c(t)        : & \left\{ c(1), c(2), \cdots, c(N) \right\} \\
        z(t)-z_d(t) : & \left\{ z(1)-z_d(1), z(2)-z_d(2), \cdots, z(N)-z_d(N) \right\} \\
        \ddot{z}(t) : & \left\{ \ddot{z}(1), \ddot{z}(2), \cdots, \ddot{z}(N) \right\} \\
        \dot{z}(t)  : & \left\{ \dot{z}(1), \dot{z}(2), \cdots, \dot{z}(N) \right\} \text{ (just for training)} \\
    \end{align*}

    % this part has been done the 18/05/2020

    Back to the main equation:
    \[
        M\ddot{z} = -K(z-z_d)-C(t)(\dot{z}-\dot{z}_d)
    \]
    \[
        \underbrace{\dot{z}}_{\text{to be estim.}} =
        -\frac{K}{M} \underbrace{\int (z-z_d)dt}_{r_1(t)}
        -\frac{1}{M} \underbrace{\int C(t)(\dot{z}-\dot{z}_d)dt}_{r_2(t)}
    \]

    We also consider this equation
    \[
        \dot{z}_d = \underbrace{\int \ddot{z}_d dt}_{r_3(t)}
    \]

    $r_1(t)$ and $r_2(t)$ are the primary regressors, directly linked to $\dot{z}(t)$. $r_3(t)$ is a secondary regressor, it can help $r_1(t)$.

    Since these regressors are obtained by integration to avoid drifting (by DC components of noise integration) we have to high-pass the inputs with high-pass filters $\left(\frac{z-1}{z-a}\right)$.

    \paragraph{Full filtering scheme} \phantom{lol}
    \begin{figure}[H]
        \centering
        \begin{tikzpicture}[node distance=2cm,auto,>=latex']
            \draw[int, dashed border] (0.5,-0.5) rectangle ++(5.5,6);
            \node[int, double border, minimum width=1.5cm, minimum height=6cm] at (8,2.5) (f) {$f(\cdot, \theta)$};
            \node[left] at (0,0) (c) {$c(t)$};
            \node[left] at (0,3) (d) {$z-z_d$};
            \node[left] at (0,5) (z) {$\ddot{z}$};
            \node[sum] at (2,0) (mult) {$\times$};
            \node[int] at (2,1.5) (d1) {$\frac{z-1}{z}$};
            \node[int] at (3.5,0) (d2) {$\frac{z-1}{z-a}$};
            \node[int] at (5,0) (d3) {$\frac{1}{z-1}$};
            \node[int] at (3.5,3) (d4) {$\frac{z-1}{z-a}$};
            \node[int] at (5,3) (d5) {$\frac{1}{z-1}$};
            \node[int] at (3.5,5) (d6) {$\frac{z-1}{z-a}$};
            \node[int] at (5,5) (d7) {$\frac{1}{z-1}$};

            \node[below] at (3,-0.7) {regressors building block};

            \draw[->] (c) -- (mult);
            \draw[->] (d1) -- (mult);
            \draw[->] (mult) -- (d2);
            \draw[->] (d2) -- (d3);
            \draw[->] (z) -- (d6);
            \draw[->] (d6) -- (d7);
            \draw[->] (d) -- (d4);
            \draw[->] (d4) -- (d5);
            \draw[->] (d) -| (d1);

            \draw[->] (d7) -- (d7-|f.west) node[pos=0.7] {$r_3(t)$};
            \draw[->] (d5) -- (d5-|f.west) node[pos=0.7] {$r_1(t)$};
            \draw[->] (d3) -- (d3-|f.west) node[pos=0.7] {$r_2(t)$};
            \draw[->] (f) -- (9.5,2.5) node[right] {$\hat{\dot{z}}$};
        \end{tikzpicture}
    \end{figure}
\end{exercise}
