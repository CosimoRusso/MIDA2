\newlecture{Sergio Savaresi}{19/05/2020}

\section{Using Simulation Error Method}

Are there alternative ways to solve gray-box system identification problems?
A commonly (and intuitive) used method is parametric identification approach based on Simulation Error Method (SEM).

\begin{figure}[H]
    \centering
    \begin{tikzpicture}[node distance=2cm,auto,>=latex']
        \node[int, align=center] at (0,0) (sys) {Model with\\some unknown\\parameters};
        \draw[<-] (sys) -- ++(-2,0) node[left] {$u(t)$};
        \draw[->] (sys) -- ++(2,0) node[right] {$y(t)$};
    \end{tikzpicture}
\end{figure}

\paragraph{Step 1} Collect data from an experiment

\begin{align*}
    \{ \tilde{u}(1), \tilde{u}(2), \dots, \tilde{u}(N) \} \\
    \{ \tilde{y}(1), \tilde{y}(2), \dots, \tilde{y}(N) \}
\end{align*}

\paragraph{Step 2} Define model structure
\[
    y(t) = \mathcal{M}(u(t), \overline{\theta}, \theta)
\]
Mathematical model (linear or non-linear) usually written from first principle equations. $\overline{\theta}$ is the set of known parameters (mass, resistance, \dots), $\theta$ is the set of unknown parameters (possibly with bounds).

\paragraph{Step 3} Performance index definition
\[
    J_N(\theta) = \frac{1}{N} \sum_{t=1}^N \left( \tilde{y}(t) - \mathcal{M}(\tilde{u}(t), \overline{\theta}, \theta) \right)^2
\]

\paragraph{Step 4} Optimization

\[
    \hat{\theta}_N = \argmin_\theta J_N(\theta)
\]

\begin{itemize}
    \item Usually no analytic expression of $J_N(\theta)$ is available.
    \item Each computation of $J_N(\theta)$ requires an entire simulation of the model from $t=1$ to $t=N$.
    \item Usually $J_N(\theta)$ is a non-quadratic and non-convex function. Iterative and randomized optimization methods must be used.
    \item It's intuitive but very computationally demanding.
\end{itemize}

\begin{figure}[H]
    \centering
    \begin{tikzpicture}[node distance=2cm,auto,>=latex']
        \draw (0,2) rectangle ++(2,2);
        \node[align=center] at (1,3) {Model of\\the system};
        \node[draw, ellipse, align=center] at (1,0) (m) {Model\\$\mathcal{M}(\theta)$};
        \node[sum] at (4,0) (sum) {};
        \node[int] at (4,-1.5) (J) {$J(\theta)$};

        \draw[<-] (m) -- (-1,0) node[left] {$\tilde{F}(t)$};
        \draw[->] (m) -- (sum) node[pos=0.8] {+} node[pos=0.5] {$\hat{y}(t)$};
        \draw[->] (4,3) -- (sum) node[pos=0.9] {-};
        \draw[->] (sum) -- (J) node[pos=0.5, right] {\footnotesize simulation error};
        \draw[->] (J) -| (m);

        \draw[<-] (0,3) -- (-1,3) node[left] {$\tilde{F}(t)$};
        \draw[->] (2,3) -- (5,3) node[right] {$\tilde{y}(t)$};
    \end{tikzpicture}
\end{figure}

Can S.E.M. be applied also to B.B. methods?

\begin{example}
    We collect data $\{ \tilde{u}(1), \tilde{u}(2), \dots, \tilde{u}(N) \}$ and $\{ \tilde{y}(1), \tilde{y}(2), \dots, \tilde{y}(N) \}$, we want to estimate from data the I/O model.

    \[
        y(t) = \frac{b_0 + b_1z^{-1}}{1+a_1z^{-1} + a_2z^{-2}}u(t-1) \qquad \theta = \begin{bmatrix}
            a_1 \\ a_2 \\ b_0 \\ b_1
        \end{bmatrix}
    \]

    In time domain $y(t) = -a_1y(t-1)-a_2y(t-2)+b_0u(t-1)+b_1u(t-2)$.

    Using P.E.M.
    \[
        \hat{y}(t|t-1) = -a_1\hat{y}(t-1)-a_2\hat{y}(t-2)+b_0\hat{u}(t-1)+b_1\hat{u}(t-2)
    \]
    \begin{align*}
        J_N(\theta) &= \frac{1}{N}\sum_{t=1}^N \left( \tilde{y}(t) - \hat{y}(t|t-1, \theta) \right)^2 \\
        &= \frac{1}{N}\sum_{t=1}^N \left( \tilde{y}(t) +a_1\tilde{y}(t-1)+a_2\tilde{y}(t-2)-b_0\tilde{u}(t-1)-b_1\tilde{u}(t-2) \right)^2 \\
    \end{align*}

    Notice that it's a quadratic formula.

    \begin{figure}[H]
        \begin{minipage}[t]{0.5\textwidth}
            \centering
            \begin{tikzpicture}[node distance=2.5cm,auto,>=latex']
                \node[int] at (1,1) (zu) {$z^{-1}$};
                \node[int] at (1,3.5) (zy) {$z^{-1}$};

                \node at (0,2) (u) {$\tilde{u}(t)$};
                \node at (0,4.5) (y) {$\tilde{y}(t)$};

                \node[int,minimum height=5cm,minimum width=1.5cm,align=center] at (3.5,2.5) (sys) {Linear\\function\\of $\theta$};

                \draw[->] (zu) -- (zu-|sys.west) node[pos=0.5] {$\scriptstyle\tilde{u}(t-1)$};
                \draw[->] (u) -- (u-|sys.west);
                \draw[->] (1,2) -- (zu);
                \draw[->] (zy) -- (zy-|sys.west) node[pos=0.5] {$\scriptstyle\tilde{y}(t-1)$};
                \draw[->] (y) -- (y-|sys.west);
                \draw[->] (1,4.5) -- (zy);
                \draw[->] (sys) -- ++(2,0) node[above] {$\scriptstyle\hat{y}(t|t-1)$};
            \end{tikzpicture}
            \caption*{P.E.M.}
        \end{minipage}
        \begin{minipage}[t]{0.4\textwidth}
            \centering
            \begin{tikzpicture}[node distance=2.5cm,auto,>=latex']
                \node[int] at (1,1) (zu) {$z^{-1}$};
                \node[int] at (1,3) (zy) {$z^{-1}$};
                \node[int] at (1,4.5) (zy2) {$z^{-1}$};

                \node at (0,2) (u) {$\tilde{u}(t)$};

                \node[int,minimum height=5cm,minimum width=1.5cm,align=center] at (3.5,2.5) (sys) {Linear\\function\\of $\theta$};

                \draw[->] (zu) -- (zu-|sys.west) node[pos=0.5] {$\scriptstyle\tilde{u}(t-1)$};
                \draw[->] (u) -- (u-|sys.west);
                \draw[->] (1,2) -- (zu);

                \draw[->] (zy) -- (zy-|sys.west) node[pos=0.5] {$\scriptstyle\hat{y}(t-2)$};
                \draw[->] (zy2) -- (zy2-|sys.west) node[pos=0.5] {$\scriptstyle\hat{y}(t-1)$};
                \draw[->] (zy2) -- (zy);
                \draw[->] (4.8,2.5) -- (4.8,5.5) -- (1,5.5) -- (zy2);

                \draw[->] (sys) -- ++(2,0) node[above] {$\scriptstyle\hat{y}(t|t-1)$};
            \end{tikzpicture}
            \caption*{S.E.M.}
        \end{minipage}
    \end{figure}

    Using S.E.M.
    \[
        \hat{y}(t|t-1) = -a_1\hat{y}(t-1)-a_2\hat{y}(t-2)+b_0\tilde{u}(t-1)+b_1\tilde{u}(t-2)
    \]
    \begin{align*}
        J_N(\theta) &= \frac{1}{N}\sum_{t=1}^N \left( \tilde{y}(t) - \hat{y}(t|t-1, \theta) \right)^2 \\
        &= \frac{1}{N}\sum_{t=1}^N \left( \tilde{y}(t) +a_1\hat{y}(t-1)+a_2\hat{y}(t-2)-b_0\tilde{u}(t-1)-b_1\tilde{u}(t-2) \right)^2 \\
    \end{align*}

    Notice that it's non-linear with respect to $\theta$.
\end{example}

P.E.M. approach looks much better, but do not forget the noise! P.E.M. is much less robust w.r.t. noise, we must include a model of the noise in the estimated model.
We use ARMAX models.

If we use ARX models:
\[
    y(t) = \frac{b_0+b_1z^{-1}}{1+a_1z^{-1}+a_2z^{-2}}u(t-1) + \frac{1}{1+a_1z^{-1}+a_2z^{-2}}e(t)
\]
\[
    \hat{y}(t|t-1) = b_0u(t-1)+b_1u(t-2) - a_1y(t-1)-a_2y(t-2)
\]

If we use ARMAX models the numerator of the T.F. for $e(t)$ is $1+c_1z^{-1}+\ldots+c_mz^{-m}$, in this case $J_N(\theta)$ is non-linear.
This leads to the same complexity of S.E.M.

The second problem of P.E.M. is high sensitivity to sampling time choice.
Remember that when we write at discrete time $y(t)$ we mean $y(t\cdot \Delta)$.

\[
    \hat{y}(t|t-1) = -a_1\tilde{y}(t-1)-a_2\tilde{y}(t-2) + b_0\tilde{u}(t-1)+b_1\tilde{u}(t-2)
\]

If $\Delta$ is very small the difference between $\tilde{y}(t)$ and $\tilde{y}(t-1)$ becomes very small.
The effect is that the P.E.M. optimization ends to provide this \emph{trivial} solution:
\[
    a_1 = -1 \qquad a_2 \rightarrow 0 \qquad b_0 \rightarrow 0 \qquad b_1 \rightarrow 0 \qquad \Rightarrow \qquad \tilde{y}(t) \approx \tilde{y}(t-1)
\]

This is a wrong model due to the fact that the recursive part of the model is using past measures of the output instead of past values of the model outputs.

\section{Conclusion}

Summary of system identification methods for I/O systems
\begin{figure}[H]
    \centering
    \begin{tikzpicture}[node distance=2cm,auto,>=latex']
        \node[int, dashed border, minimum width=1.5cm, minimum height=1.5cm] at (0,0) (sys) {};
        \draw[<-] (sys) -- ++(-1.5,0) node[left] {$u(t)$};
        \draw[->] (sys) -- ++(1.5,0) node[right] {$y(t)$};
    \end{tikzpicture}
\end{figure}

\begin{itemize}
    \item Collect a dataset for training (if needed)
    \item Choose a model domain (linear static/non-linear static/linear dynamic/non-linear dynamic), using gray-box or black-box
    \item Estimation method: constructive (4SID), parametric (P.E.M. or S.E.M.) or filtering (state extension of K.F.)
\end{itemize}

Better black-box for system identification and software-sensing or white box?

It depends on the goals and type of applications.

\begin{itemize}
    \item Black box is very general and very flexible, make maximum use of data and no or little need of domain knowhow
    \item White box is very useful when you are the system-designer (not only the control algorithm designer), can provide more insight in the system.
    \item Gray box can sometimes be obtained by hybrid systems (part is black-box and part is white-box).
\end{itemize}
