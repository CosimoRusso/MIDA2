\newlecture{Sergio Savaresi}{21/04/2020}

\begin{figure}[H]
    \centering
    \begin{tikzpicture}[node distance=2.5cm,auto,>=latex']
        \node (n1) {\#1};
        \node (n2) [below of=n1, xshift=-1.8cm] {\#2};
        \node (n3) [below of=n1, xshift= 1.8cm] {\#3};

        \draw[->] (n1) edge[bend right] (n2);
        \draw[->] (n2) edge[bend right] (n3);
        \draw[->] (n1) edge[bend left] (n3);
        \draw[->, dashed] (n2) edge[bend right=20] (n1);
        \draw[->, dotted] (n3) edge[bend right=20] (n2);
        \draw[->, dashed, red] (n3) edge[bend left=20] node {?} (n1);
    \end{tikzpicture}
\end{figure}

\section{Fundamental concepts of observability and controllability}

\[
    \begin{cases}
        x(t+1) = Fx(t) + Gu(t) \\
        y(t) = Hx(t)
    \end{cases}
\]

\subsection{Fully Observable}
The system is fully observable (from the output) if and only if the observability matrix is full rank:
\[
    O = \begin{bmatrix}
        H \\
        HF \\
        \vdots \\
        HF^{n-1}
    \end{bmatrix}
    \qquad
    \rank O = n
\]

\subsection{Fully Controllable}
The system is fully controllable (from the input) if and only if the controllability matrix is full rank:
\[
    R = \begin{bmatrix}
        G & FG & \cdots & F^{n-1}G
    \end{bmatrix}
    \qquad
    \rank R = n
\]

$R$ is also called \emph{reachability} matrix.

\begin{remark}
    \begin{description}
        \item[Observability] we can observe the state from the output sensors.
        \item[Controllability] we can control/move/influence the state using the input signal.
    \end{description}
\end{remark}

\begin{example}
    \begin{align*}
        \begin{cases}
            x_1(t+1) = \frac{1}{2} x_1(t) + u(t) \\
            x_2(t+1) = \frac{1}{3}x_2(t) \\
            y(t) = \frac{1}{4}x_1(t)
        \end{cases}
        \qquad
        F = \begin{bmatrix}
            \frac{1}{2} & 0 \\
            0 & \frac{1}{3}
        \end{bmatrix}
        \qquad
        H = \begin{bmatrix}
            \frac{1}{4} & 0
        \end{bmatrix}
        \qquad
        G = \begin{bmatrix}
            0 \\
            1
        \end{bmatrix}
    \end{align*}

    \[
        O = \begin{bmatrix}
            H \\
            HF
        \end{bmatrix} = \begin{bmatrix}
            \frac{1}{4} & 0 \\
            \frac{1}{8} & 0
        \end{bmatrix}
        \qquad
        \rank O = 1 < n = 2
        \quad\implies\quad \text{not fully observable}
    \]

    \[
        R = \begin{bmatrix}
            G & FG
        \end{bmatrix} = \begin{bmatrix}
            0 & 0 \\
            1 & \frac{1}{3}
        \end{bmatrix}
        \qquad
        \rank R = 1 < n = 2
        \quad\implies\quad \text{not fully controllable}
    \]
\end{example}

\begin{remark}[4 sub-systems]
    Each system can be internally seen as 4 sub-systems as follows:

    \begin{figure}[H]
        \centering
        \begin{tikzpicture}[node distance=1.5cm,auto,>=latex']
            \node[int, align=center] (noc) {NO\\C};
            \node[int, align=center, below of=noc] (onc) {O\\NC};
            \node[int, align=center, below of=onc] (oc) {O\\C};
            \node[int, align=center, below of=oc] (nonc) {NO\\NC};

            \node at (-5cm,-1.5cm) (u) {$u$};
            \node[coordinate] at (-3cm,-1.5cm) (input) {};
            \node at (5cm,-2.5cm) (y) {$y$};
            \node[coordinate] at (3cm,-2.5cm) (output) {};
            \node[right of=noc] (noc_out) {};
            \node[right of=nonc] (nonc_out) {};
            \node[left of=onc] (onc_in) {};

            \draw[-{Rays[n=4]}] (noc) -- (noc_out);
            \draw[-{Rays[n=4]}] (nonc) -- (nonc_out);
            \draw[{Rays[n=4]}-] (onc_in) -- (onc);
            \draw (input) edge[->, bend left=10] (noc);
            \draw[line width=0.5mm] (input) edge[->, bend right=10] (oc);
            \draw[line width=0.5mm] (u) edge (input);
            \draw (onc) edge[bend left=10] (output);
            \draw[line width=0.5mm] (oc) edge[bend right=10] (output);
            \draw[line width=0.5mm] (output) edge[->] (y);

            \draw[draw=black] (-4cm,-5.5cm) rectangle ++(8cm,6.5cm);
        \end{tikzpicture}
    \end{figure}

    Which externally is equivalent to a systems like this:
    \begin{figure}[H]
        \centering
        \begin{tikzpicture}[node distance=1.5cm,auto,>=latex']
            \node[int, align=center] (oc) {O\\C};
            \node[left of=oc] (in) {$u$};
            \node[right of=oc] (out) {$y$};

            \draw[->] (in) -- (oc);
            \draw[->] (oc) -- (out);
        \end{tikzpicture}
    \end{figure}
\end{remark}

\section{Hankel matrix of order n}

Starting from $IR = \{\omega(1), \omega(2), \ldots, \omega(N)\}$ we can build the Hankel matrix of order $n$.

\[
    H_n = \begin{bmatrix}
        \omega(1) & \omega(2) & \omega(3) & \cdots & \omega(n) \\
        \omega(2) & \omega(3) & \omega(4) & \cdots & \omega(n+1) \\
        \omega(3) & \omega(4) & \omega(5) & \cdots & \omega(n+2) \\
        \vdots    & \vdots    & \vdots    & \ddots & \vdots \\
        \omega(n) & \omega(n+1) & \omega(n+2) & \ldots & \omega(2n-1)
    \end{bmatrix}
\]

\textbf{Note} that we need IR up to time $2n-1$.

\textbf{Note} it starts from $\omega(1)$ and not $\omega(0)$.

\textbf{Note} the anti-diagonals all have the same element repeated.

We know that $\omega(t) = \begin{cases}
    0 &\quad \text{if } t = 0 \\
    HF^{t-1}G &\quad \text{if } t > 0
\end{cases}$ therefore $H_n$ can be rewritten as

\[
    H_n = \begin{bmatrix}
        HG     & HFG    & HF^2G  & \cdots & HF^{n-1}G \\
        \vdots & \ddots &        &        & \vdots \\
        \vdots &        & \ddots &        & \vdots \\
        \vdots &        &        & \ddots & \vdots \\
        HF^{n-1}G & \cdots & \cdots & \cdots & HF^{2n-2}G
    \end{bmatrix} = \begin{bmatrix}
        H \\
        HF \\
        \vdots \\
        HF^{n-1}
    \end{bmatrix} \cdot \begin{bmatrix}
        G & FG & \cdots & F^{n-1}G
    \end{bmatrix} = O \cdot R
\]

Where $O$ is the observability matrix and $R$ is the reachability matrix.

\section{Algorithm to obtain $\hat{F}$, $\hat{G}$, $\hat{H}$ from a noise-free measured IR}

\paragraph{Step 1} Build the Hankel matrix in increasing order and each time compute the rank of the matrix.

\[
    H_1 = \begin{bmatrix}
        \omega(1)
    \end{bmatrix}
    \qquad
    H_2 = \begin{bmatrix}
        \omega(1) & \omega(2) \\
        \omega(2) & \omega(3)
    \end{bmatrix}
    \qquad
    H_3 = \ldots
    \qquad
    \cdots
    \qquad
    H_n = \ldots
\]

Suppose that $\rank H_n = n$ and $\rank H_{n+1} = n$.
With this procedure we know the order of the system.

\paragraph{Step 2} Take $H_{n+1}$ (with $\rank H_{n+1} = n$), factorize it in two rectangular matrices of size $(n+1) \times n$ and $n \times (n+1)$.

\[
    H_{n+1} = \begin{bmatrix}
        \text{extended} \\
        \text{observability} \\
        \text{matrix} \\
        O_{n+1}
    \end{bmatrix} \cdot \begin{bmatrix}
        \text{extended} \\
        \text{controllability} \\
        \text{matrix} \\
        R_{n+1}
    \end{bmatrix}
\]

Where $O_{n+1} = \begin{bmatrix}
    H \\ HF \\ \vdots \\ HF^n
\end{bmatrix}$ of size $(n+1)\times n$ and $R_{n+1} = \begin{bmatrix}
    G & FG & \cdots & F^nG
\end{bmatrix}$ of size $n\times (n+1)$.

\paragraph{Step 3} $H$, $F$, $G$ estimation.

Using $O_{n+1}$ and $R_{n+1}$ we can easily find:
\begin{align*}
    \hat{G} = R_{n+1}(\texttt{:;1}) & \quad\text{first column} \\
    \hat{H} = O_{n+1}(\texttt{1;:}) & \quad\text{first row} \\
\end{align*}

Define $O_1$ as $O_{n+1}$ without the last row, and $O_2$ as $O_{n+1}$ without the first row.
$O_1$ and $O_2$ are $n\times n$ matrices.
This is called \emph{shift-invariance} property.

\textbf{Note} that $O_1$ is invertible since it's an observability matrix and the subsystem is fully observable.
Moreover $O_1F = O_2$, so $\hat{F} = O_1^{-1}O_2$.

In conclusion in a simple and constructive way we have estimated a State Space model of the system $\{\hat{H}, \hat{G}, \hat{F}\}$ starting from measured IR, using only $2n+1$ samples of IR.

\begin{remark}
    If the measurement is noisy all this process is useless.
\end{remark}

\section{Real problem}

The measurements of IR are noisy: $\tilde{\omega}(t) = \omega(t) + \eta(t)$.

\begin{figure}[H]
    \begin{minipage}[t]{0.5\textwidth}
        \centering
        \begin{tikzpicture}[node distance=2.5cm,auto,>=latex']
            \draw[->] (-0.5,0) -- (3,0) node[right] {$t$};
            \draw[->] (0,-1) -- (0,2) node[left] {$u(t)$};
            \draw[domain=-0.3:0,smooth,variable=\x,red] plot ({\x},{0});
            \draw[domain=0:0.2,smooth,variable=\x,red] plot ({\x},{1.5});
            \draw[domain=0.2:2.5,smooth,variable=\x,red] plot ({\x},{0});
        \end{tikzpicture}
        \caption*{Input}
    \end{minipage}
    \begin{minipage}[t]{0.5\textwidth}
        \centering
        \begin{tikzpicture}[node distance=2.5cm,auto,>=latex']
            \draw[->] (-0.5,0) -- (3,0) node[right] {$t$};
            \draw[->] (0,-1) -- (0,2) node[left] {$y(t)$};
            \draw[domain=0:2.5,smooth,variable=\x,red] plot ({\x},{cos(\x*180/3.14*5)*e^(-\x)});
            \draw[samples=100, domain=0:2.5,smooth,variable=\x,blue] plot ({\x},{cos(\x*180/3.14*5)*e^(-\x)+rand/4});
        \end{tikzpicture}
        \caption*{Output}
    \end{minipage}
\end{figure}

\section{4SID procedure (with noise)}

\paragraph{Step 1} Build the Hankel matrix from data using ``one-shot'' all the available $N$ data points.

\[
    \tilde{H}_{qd} = \begin{bmatrix}
        \tilde{\omega}(1) & \tilde{\omega}(2) & \cdots & \tilde{\omega}(d) \\
        \tilde{\omega}(2) & \tilde{\omega}(3) & \cdots & \tilde{\omega}(d+1) \\
        \vdots            & \vdots            & \ddots & \vdots \\
        \tilde{\omega}(q) & \tilde{\omega}(q+1) & \cdots & \tilde{\omega}(q+d-1) \\
    \end{bmatrix}
\]

$\tilde{H}_{qd}$ is a $q\times d$ matrix. \textbf{Note} that $q+d-1$ must equal to $N$ so that we use all the data set.

\begin{remark}[Choice of $q$ and $d$]
    Hypothesis: $q<d$ so that $q+d-1=N$, so $q=N+1-d$.
    \begin{figure}[H]
        \centering
        \begin{tikzpicture}[node distance=2.5cm,auto,>=latex']
            \draw[->] (-0.5,0) -- (3,0) node[right] {$d$};
            \draw[->] (0,-0.5) -- (0,3) node[left] {$q$};
            \draw[domain=0.3:2.7,smooth,variable=\x,black] plot ({\x},{3-\x});
            \draw[line width=0.6mm,domain=1.5:1.8,smooth,variable=\x,green] plot ({\x},{3-\x});
            \draw[line width=0.6mm,domain=2.3:2.7,smooth,variable=\x,blue] plot ({\x},{3-\x});
            \draw[dotted] (0,0) -- (1.5,1.5);

            \node at (2.3,1.5) {$q \approx d$};
            \node at (3.1,0.7) {$q \ll d$};

            \node[below] at (0.3,0) {$1$};
            \node[below] at (2.7,0) {$N$};
            \node[left] at (0,0.3) {$1$};
            \node[left] at (0,2.7) {$N$};
        \end{tikzpicture}
    \end{figure}

    If $q \approx d$ the method has better accuracy.
    If $q \ll d$ it's computationally less intensive.

    If $0.6d < q < d$ we get to a good enough result.
\end{remark}
