\chapter{Minimum Variance Control}
\newlecture{Sergio Savaresi}{25/05/2020}

Design and analysis of feedback systems.
It's not system identification nor software sensing.

\begin{itemize}
    \item Control design is the main motivation to system identification and software sensing.
    \item MVC is based on \emph{mathematics} of system identification and software sensing (prediction theory).
\end{itemize}

\paragraph{Setup of the problem} Consider a generic ARMAX model:
\[
    y(t) = \frac{B(z)}{A(z)}u(t-k) + \frac{C(z)}{A(z)}e(t) \qquad e(t) \sim WN(0, \lambda^2)
\]
\begin{align*}
    B(z) &= b_0 + b_1z^{-1} + \dots + b_pz^{-p} \\
    A(z) &= 1   + a_1z^{-1} + \dots + a_mz^{-m} \\
    C(z) &= 1   + c_1z^{-1} + \dots + c_nz^{-n}
\end{align*}

Assumptions:
\begin{itemize}
    \item $\frac{C(z)}{A(z)}$ is in canonical form
    \item $b_0\ne 0$ ($k$ is the actual delay)
    \item $\frac{B(z)}{A(z)}$ is \emph{minimum phase}
\end{itemize}

\begin{remark}
    $\frac{B(z)}{A(z)}$ is said to be \emph{minimum phase} if all the roots of $B(z)$ are strictly inside the unit circle.

    \begin{figure}[H]
        \centering
        \begin{tikzpicture}[node distance=2cm,auto,>=latex']
            \draw[->] (0,4) -- (0,7) node[left] {$u(t)$};
            \draw[->] (0,4) -- (5,4) node[below] {$t$};
            \draw[->] (0,0) -- (0,3) node[left] {$y(t)$};
            \draw[->] (0,0) -- (5,0) node[below] {$t$};

            \draw[line width=0.4mm] (0,4) -- (1,4) -- (1,6.5) -- (5,6.5);
            \draw[dotted] (1,0) -- (1,4);
            \draw[dotted] (0,2.5) -- (5,2.5) node[right] {\footnotesize steady state};

            \draw[domain=1:4.5,smooth,variable=\x,green,samples=70] plot ({\x},{2.5-2.5*(1-(\x-1)/3.5)^5});
            \draw[domain=1:4.5,smooth,variable=\x,green,samples=70] plot ({\x},{2.5-2.5*(1-(\x-1)/3.5)^5+sin(\x*180)/\x});
            \draw[domain=1:4.5,smooth,variable=\x,red,samples=70] plot ({\x},{2.5-2.5*(1-(\x-1)/3.5)^5+3*sin(\x*180)/(\x^2)});

            \node[red,right] at (1.5,-0.5) {\footnotesize not minimum-phase};
            \node[green,right] at (2,1.5) {\footnotesize minimum-phase};
        \end{tikzpicture}
    \end{figure}

    At the very beginning the response of a non-minimum phase system goes to the opposite direction w.r.t. the final value.


    Intuitively it's very difficult to control non-minimum phase systems.
    You can take the wrong decision if you react immediately.

    Also for human it's difficult, for example \emph{steer to roll} dynamics in a bicycle: if you want to steer left, you must first steer a little to the right and then turn left.

    Design of controller for non-minimum phase is difficult and requires special design techniques (no MVC but generalized MVC).
\end{remark}

The problem we wish to solve is optimal tracking of the desired behavior of output:
\begin{figure}[H]
    \centering
    \begin{tikzpicture}[node distance=2cm,auto,>=latex']
        \node[int,ellipse,align=center] at (-1,0) (cont) {controller\\algorithm};
        \node[int] at (2.5,0) (ba) {$\frac{B(z)}{A(z)}$};
        \node[int] at (4,1.5) (ca) {$\frac{C(z)}{A(z)}$};
        \node[sum] at (4,0) (sum) {};

        \draw[dotted] (0.8,-1) rectangle (5,3.2) node[right] {system};
        \draw[->] (cont) -- (ba) node[pos=0.7] {$u(t)$};
        \draw[->] (ba) -- (sum);
        \draw[->] (ca) -- (sum);
        \draw[<-] (cont) -- ++(-2,0) node[left] {$y^0(t)$};
        \draw[<-] (ca) -- ++(0,1) node[above] {$e(t)$};
        \draw[->] (sum) -- ++(2,0) node[right] {$y(t)$};
        \draw[->] (5.5,0) -- (5.5,-1.5) -- (-1,-1.5) -- (cont);
    \end{tikzpicture}
\end{figure}

In a more formal way MVC tries to minimize this performance index:
\[
    J = E\left[ (y(t) - y^0(t))^2 \right]
\]
It's the variance of the tracking error, that's why it's called Minimum Variance Control.

Some additional (small) technical assumptions:
\begin{itemize}
    \item $y^0(t)$ and $e(t)$ are not correlated (usually fulfilled)
    \item We assume that $y^0(t)$ is known only up to time $t$ (present time), we have no preview of the future desired $y^0(t)$ ($y^0(t)$ is totally unpredictable).
\end{itemize}

\begin{remark}
    There are 2 sub-classes of control problems:
    \begin{itemize}
        \item When $y^0(t)$ is constant or step-wise (regulation problem)
        \item When $y^0(t)$ is varying (tracking problem)
    \end{itemize}

    \begin{figure}[H]
        \centering
        \begin{minipage}[t]{0.48\textwidth}
            \centering
            \begin{tikzpicture}[
                node distance=2cm,auto,>=latex',
                declare function={
                    f(\x) =  (\x < 0.5) * 1 +
                             (\x >= 0.5) * (\x < 2) * 2 +
                             (\x >= 2) * (\x < 3) * 3 +
                             (\x >= 3) * (\x < 4) * 1.5 +
                             (\x >= 4) * 1;
                    f2(\x) = (f(\x-0.5) + (f(\x) - f(\x-0.5)) / 720 +
                             f(\x-0.4) + (f(\x) - f(\x-0.4)) / 120 +
                             f(\x-0.3) + (f(\x) - f(\x-0.3)) / 24 +
                             f(\x-0.2) + (f(\x) - f(\x-0.2)) / 6 +
                             f(\x-0.1) + (f(\x) - f(\x-0.1)) / 2) / 5 +
                             rand/8;
                }
            ]
                \draw[->] (0,0) -- (0,3) node[above] {$y(t)$};
                \draw[->] (0,0) -- (5,0) node[below] {$t$};
                \draw[domain=0:5,variable=\x,blue,samples=100] plot ({\x},{f(\x)});
                \draw[domain=0:5,variable=\x,red,smooth,samples=100] plot ({\x},{f2(\x)});
            \end{tikzpicture}
            \caption*{Regulation problem}
        \end{minipage}
        \begin{minipage}[t]{0.48\textwidth}
            \centering
            \begin{tikzpicture}[
                node distance=2cm,auto,>=latex',
                declare function={
                    f(\x) =  (sin(\x*180)/2+sin(\x*270)/2)+1.5;
                    f2(\x) = (f(\x-0.3) + (f(\x) - f(\x-0.3)) / 24 +
                             f(\x-0.2) + (f(\x) - f(\x-0.2)) / 6 +
                             f(\x-0.1) + (f(\x) - f(\x-0.1)) / 2) / 3 +
                             rand/8;
                }
            ]
                \draw[->] (0,0) -- (0,3) node[above] {$y(t)$};
                \draw[->] (0,0) -- (5,0) node[below] {$t$};
                \draw[domain=0:5,variable=\x,blue,samples=70] plot ({\x},{f(\x)});
                \draw[domain=0:5,variable=\x,red,smooth,samples=50] plot ({\x},{f2(\x)});
            \end{tikzpicture}
            \caption*{Tracking problem}
        \end{minipage}
    \end{figure}
\end{remark}

Bottom-up way of presenting M.V.C.

\paragraph{Simplified problem \#1}
\[
    S: y(t) = ay(t-1) + b_0u(t-1) + b_1u(t-2) \qquad y(t) = \frac{b_0+b_1z^{-1}}{1-az^{-1}}u(t-1)
\]

We assume that $y^0(t)=\overline{y}^0$ (regulation problem) and the system is noise-free.
\begin{itemize}
    \item $b_0\ne 0$
    \item Root of numerator must be inside the unit circle
\end{itemize}

To design the minimum variance controller we must minimize the performance index:
\[
    J = E\left[ (y(t) - y^0(t))^2 \right]
\]
There is no noise so we can remove the expected value
\begin{align*}
    J &= \left( y(t) - y^0(t) \right)^2 = \left( y(t) - \overline{y}^0 \right)^2 = \left( ay(t-1)+b_0u(t-1)+b_1u(t-2) - \overline{y}^0 \right)^2 = \\
    &= \left( ay(t) + b_0u(t) + b_1u(t-1)-\overline{y}^0 \right)^2 \\
    \frac{\partial J}{\partial u(t)} &= 2\left( ay(t)+b_0u(t)+b_1u(t-1)-\overline{y}^0 \right)\left(b_0\right)
\end{align*}

Why the derivative is just $b_0$? We are at present time $t$ and at time $t$ the control algorithm must take a decision on the value of $u(t)$.
At time $t$, $y(t)$, $y(t-1)$, \dots, $u(t-1)$, $u(t-2)$, \dots{} are no longer variables but numbers.

\[
    \frac{\partial J}{\partial u(t)} = 0 \qquad ay(t)+b_0u(t)+b_1u(t-1)-\overline{y}^0 = 0 \\
\]
\[
    u(t) = \left( \overline{y}^0  - ay(t)\right)\frac{1}{b_0+b_1z^{-1}}
\]
\begin{figure}[H]
    \centering
    \begin{tikzpicture}[node distance=2cm,auto,>=latex']
        \node[sum] at (0,0) (sum) {};
        \node[int] at (1.5,0) (b1) {$\frac{1}{b_0+b_1z^{-1}}$};
        \node[int] at (5,0) (b2) {$z^{-1}\frac{b_0+b_1z^{-1}}{1-az^{-1}}$};
        \node[int] at (3,-1.5) (b3) {$a$};

        \draw[->] (sum) -- (b1);
        \draw[->] (b1) -- (b2) node[pos=0.5] {$u(t)$};
        \draw[->] (b3) -| (sum) node[pos=0.9] {-};
        \draw[<-] (sum) -- ++(-1.5,0) node[left] {$\overline{y}^0$} node[pos=0.2] {+};
        \draw[->] (b2) -- ++(2.5,0) node[right] {$y(t)$};
        \draw[->] (6.5,0) |- (b3);
    \end{tikzpicture}
\end{figure}

\paragraph{Simplified problem \#2}
\[
    S: y(t) = ay(t-1) + b_0u(t-1) + b_1u(t-2) + e(t) \qquad e(t) \sim WN(0, \lambda^2)
\]

The reference variable is $y^0(t)$ (tracking problem).

The performance index is
\[
    J = E\left[ (y(t) - y^0(t))^2 \right]
\]

The fundamental trick to solve this problem is to re-write $y(t)$ as:
\[
    y(t) = \hat{y}(t|t-1) + \epsilon(t)
\]

Since $k=1$ we know that $\epsilon(t) = e(t)$, so $y(t) = \hat{y}(t|t-1)+e(t)$.
\begin{align*}
    J &= E\left[ \left(\hat{y}(t|t-1)+e(t) - y^0(t)\right)^2 \right] \\
    &= E\left[   \left((\hat{y}(t|t-1)-y^0(t)) +e(t)\right)^2 \right] \\
    &= E\left[ \left(\hat{y}(t|t-1)-y^0(t)\right)^2 \right] + E\left[e(t)^2\right] + \cancel{2E\left[e(t)\left( \hat{y}(t|t-1)-y^0(t) \right)\right]} \\
\end{align*}

Notice that
\[
    \argmin_{u(t)} \left\{ E\left[ \left(\hat{y}(t|t-1)-y^0(t)\right)^2 \right] + \lambda^2 \right\} = \argmin_{u(t)} \left\{ E\left[ \left(\hat{y}(t|t-1)-y^0(t)\right)^2 \right] \right\}
\]
The best result is when $\hat{y}(t|t-1)=y^0(t)$, we can force this relationship.

Now we must compute the 1-step predictor of the system:
\[
    S: y(t) = \frac{b_0+b_1z^{-1}}{1-az^{-1}}u(t-1) + \frac{1}{1-az^{-1}}e(t)
\]
Note that this is an $ARMAX(1,0,1+1)=ARX(1,2)$.
\[
    k=1 \qquad B(z) = b_0+b_1z^{-1} \qquad A(z)=1-az^{-1} \qquad C(z) = 1
\]

General solution for 1-step prediction of ARMAX:
\[
    \hat{y}(t|t-1) = \frac{B(z)}{C(z)}u(t-1) + \frac{C(z)-A(z)}{C(z)}y(t)
\]
If we apply this formula we obtain:
\[
    \hat{y}(t|t-1) = \frac{b_0+b_1z^{-1}}{1}u(t-1) + \frac{1-1+az^{-1}}{1}y(t) = (b_0+b_1z^{-1})u(t-1)+ay(t-1)
\]

Now we can impose that $\hat{y}(t|t-1)=y^0(t)$
\[
    b_0u(t) + b_1u(t-1) + ay(t) = y^0(t+1) \qquad u(t) = \left( y^0(t+1) - ay(t) \right)\frac{1}{b_0+b_1z^{-1}}
\]
But we don't have $y^0(t+1)$, so we use $y^0(t)$.
\[
    u(t) = \left( y^0(t) - ay(t) \right)\frac{1}{b_0+b_1z^{-1}}
\]
\begin{figure}[H]
    \centering
    \begin{tikzpicture}[node distance=2cm,auto,>=latex']
        \node[sum] at (0,0) (sum) {};
        \node[int] at (1.5,0) (b1) {$\frac{1}{b_0+b_1z^{-1}}$};
        \node[int] at (5,0) (b2) {$z^{-1}\frac{b_0+b_1z^{-1}}{1-az^{-1}}$};
        \node[int] at (3,-1.5) (b3) {$a$};
        \node[int] at (7,1.5) (b4) {$\frac{1}{1-az^{-1}}$};
        \node[sum] at (7,0) (sum2) {};

        \draw[->] (sum) -- (b1);
        \draw[->] (b1) -- (b2) node[pos=0.5] {$u(t)$};
        \draw[->] (b3) -| (sum) node[pos=0.9] {-};
        \draw[<-] (sum) -- ++(-1.5,0) node[left] {$y^0(t)$} node[pos=0.2] {+};
        \draw[->] (sum2) -- ++(1.5,0) node[right] {$y(t)$};
        \draw[->] (b2) -- (sum2);
        \draw[->] (b4) -- (sum2);
        \draw[<-] (b4) -- ++(0,1) node[above] {$e(t)$};
        \draw[->] (8,0) |- (b3);
    \end{tikzpicture}
\end{figure}
