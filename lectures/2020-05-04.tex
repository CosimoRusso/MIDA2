\newlecture{Sergio Savaresi}{04/05/2020}

\paragraph{Key questions for software sensing}

\begin{itemize}
    \item Is software-sensing feasible? Test is the observability of the states from measured outputs.
    \item Quality of estimation error (\emph{noise of measurement} for the software sensor).
\end{itemize}

\begin{example}[Slip estimation for ABS/traction control]
    \begin{figure}[H]
        \centering
        \begin{tikzpicture}[node distance=1.5cm,auto,>=latex']
            \draw (0.5,0) -- (0,0) -- (0,0.5) -- (1,0.5) -- (1.5,1) -- (2.5,1) -- (2.8,0.5) -- (3.3,0.5) -- (3.3,0) -- (3.1,0);
            \draw (1.1,0) -- (2.5,0);
            \draw (0.8,0) circle (0.3);
            \draw (2.8,0) circle (0.3);

            \draw[->] (-0.1,0.5) -- (-0.6,0.5) node[left] {$v$};
            \draw[->] (0.8,0) -- (0.8,-0.3) node[below] {$r$};
            \draw[->] (1.146,0.2) arc[radius=0.4, start angle=30, end angle=90];
            \node at (1.3,0.5) {$\omega$};
        \end{tikzpicture}
    \end{figure}

    \paragraph{Problem} Estimation of $v$.

    Measure of $v$ can be done with:
    \begin{itemize}
        \item Optical sensor
        \item GPS
    \end{itemize}
    Both have a problem of availability (not guaranteed). Physical sensing is not an option for industrial production.

    Intuitive solution: install a longitudinal accelerometer ($a_x$) and integrate.
    \begin{figure}[H]
        \centering
        \begin{tikzpicture}[node distance=1.5cm,auto,>=latex']
            \draw (0.5,0) -- (0,0) -- (0,0.5) -- (1,0.5) -- (1.5,1) -- (2.5,1) -- (2.8,0.5) -- (3.3,0.5) -- (3.3,0) -- (3.1,0);
            \draw (1.1,0) -- (2.5,0);
            \draw (0.8,0) circle (0.3);
            \draw (2.8,0) circle (0.3);

            \draw[->] (2.3,0.5) -- (1.7,0.5) node[above] {$a_x$};
        \end{tikzpicture}
    \end{figure}
    \[
        \hat{v} = \int a_x(t) dt \qquad <=> \qquad a_x(t) -> 1/s -> \hat{V}(t)
    \]

    In discrete time domain: discretization using approximation of derivative (Eulero forward method)
    \[
        \frac{d}{dt} v(t) = a_x(t) \qquad \frac{dv(t)}{dt} \approx \frac{v(t+1)-v(t)}{\Delta T_s} = a_x(t)
    \]

    Where $\Delta T_s$ is the sampling interval (e.g. 10ms).

    \[
        \hat{v}(t) = \hat{v}(t-1) + \Delta T_s a_x(t-1)
    \]

    \begin{figure}[H]
        \centering
        \begin{tikzpicture}[node distance=2.5cm,auto,>=latex']
            \node [int] (sys) {$\frac{\Delta T_s}{1-z^{-1}}$};
            \node (in) [left of=sys, node distance=2cm] {};
            \node (end) [right of=sys, node distance=2cm]{};

            \draw[->] (in) edge node {$a_x(t)$} (sys);
            \draw[->] (sys) edge node {$\hat{v}(t)$} (end);
        \end{tikzpicture}
    \end{figure}

    Unfortunately the measured signal is not $a_x(t)$ but $a_x(t)+d_{a_x}(t)$.

    \begin{figure}[H]
        \centering
        \begin{tikzpicture}[node distance=2.5cm,auto,>=latex']
            \begin{axis}[axis lines=none,ymax=6]
                \addplot[color=blue,smooth]
                    coordinates {
                        (0.00,2.35)(0.40,2.87)(0.80,2.99)(1.20,2.75)(1.60,2.83)(2.00,2.86)(2.40,3.19)(2.80,2.76)(3.20,2.45)(3.60,2.85)(4.00,2.55)(4.40,2.50)(4.80,2.64)(5.20,2.72)(5.60,3.26)(6.00,3.68)(6.40,3.95)(6.80,3.9)(7.20,4.00)(7.60,3.51)
                    };
                \addplot[color=red,smooth]
                    coordinates {
                        (0.00,2.35)(0.40,2.91)(0.80,3.07)(1.20,2.87)(1.60,2.99)(2.00,3.06)(2.40,3.43)(2.80,3.04)(3.20,2.77)(3.60,3.21)(4.00,2.95)(4.40,2.94)(4.80,3.12)(5.20,3.24)(5.60,3.82)(6.00,4.28)(6.40,4.59)(6.80,5.17)(7.20,5.0)(7.60,5.27)
                    };
                \draw[->] (-0.5,0) -- (800,0) node[right] {$t$};
                \draw[->] (0.5,-0.5) -- (0.5,300) node[left] {};
            \end{axis}
        \end{tikzpicture}
    \end{figure}

    Integrating noise generates a \emph{drift}. Integrator is not an asymptotic stable system.

    \begin{figure}[H]
        \centering
        \begin{tikzpicture}[node distance=2.5cm,auto,>=latex']
            \node [int] (sys) {$\frac{\Delta T_s}{1-z^{-1}}$};
            \node [sum] (in) [left of=sys, node distance=1.5cm] {};
            \node (in1) [above of=in, node distance=1cm] {$d_{a_x}(t)$};
            \node (in2) [below of=in, node distance=1cm] {$a_x(t)$};
            \node (end) [right of=sys, node distance=2cm]{};

            \draw[->] (in) -- (sys);
            \draw[->] (in1) -- (in) node[left, pos=0.8] {$+$};
            \draw[->] (in2) -- (in) node[left, pos=0.8] {$+$};
            \draw[->] (sys) edge node {$\hat{v}(t)$} (end);
        \end{tikzpicture}
    \end{figure}

    \paragraph{Solution} Use a Kalman Filter.

    \begin{figure}[H]
        \centering
        \begin{tikzpicture}[node distance=2.5cm,auto,>=latex']
            \node [int, align=center, minimum width=3cm] (sys) {car};
            \node [int, align=center, minimum width=3cm, below of=sys] (algo) {Kalman Filter};
            \node (algoout) [right of=algo, node distance=3cm] {$\hat{v}(t)$};

            \draw[->,transform canvas={xshift=-1.2cm}] (sys) -- (algo) node[pos=0.5] {$\omega_1$};
            \draw[->,transform canvas={xshift=-0.6cm}] (sys) -- (algo) node[pos=0.5] {$\omega_2$};
            \draw[->] (sys) -- (algo) node[pos=0.5] {$\omega_3$};
            \draw[->,transform canvas={xshift=0.6cm}] (sys) -- (algo) node[pos=0.5] {$\omega_4$};
            \draw[->,transform canvas={xshift=1.2cm}] (sys) -- (algo) node[pos=0.5] {$a_x$};

            \draw[->] (algo) -- (algoout);
        \end{tikzpicture}
    \end{figure}
\end{example}

\begin{example}[State of charge estimation of a battery]
    \begin{figure}[H]
        \centering
        \begin{tikzpicture}[node distance=2.5cm,auto,>=latex']
            \draw (0,0) rectangle ++(1.5cm,3cm);
            \draw (0.6cm,3cm) rectangle ++(0.3,0.15);

            \draw [pattern=north west lines, pattern color=green] (0,0) rectangle (1.5,2);

            \draw [decorate,decoration={brace,amplitude=10pt}] (0,0) -- (0,3) node [black,midway,align=right,xshift=-0.3cm] {100\%};
            \draw [decorate,decoration={brace,amplitude=10pt}] (1.5,2) -- (1.5,0) node [black,midway,align=right,xshift=0.3cm] {SoC};

            \node [int, align=center] (sys) at (6,1.5) {SoC\\internal\\state};
            \node [left of=sys, node distance=2cm] (in1) {$i(t)$};
            \node [above of=sys, node distance=2cm] (in2) {$T(t)$};
            \node [right of=sys, node distance=2cm] (out) {$v(t)$};

            \draw[->] (in1) -- (sys);
            \draw[->] (in2) -- (sys);
            \draw[->] (sys) -- (out);
        \end{tikzpicture}
    \end{figure}
    \[
        \text{SoC}(t) = 1 - \frac{\int i(t)dt}{I} \qquad 0 \le \text{SoC} \le 1
    \]
    Where $I$ is the total amount of \emph{current} that can be extracted by the user of the battery.
    This solution is not feasible since it integrates the noise on $i(t)$.
\end{example}

\section{Kalman Filter on Basic Systems}

\begin{itemize}
    \item No external inputs ($\cancel{Gu(t)}$): time series
    \item Linear systems
    \item Time invariant systems
\end{itemize}

After the description of the solution of Kalman Filter for the basic system, we make the extensions to a more general system, in particular with $Gu(t)$.

\subsection{Detailed description of Basic System}

\[
    S: \begin{cases}
        x(t+1) = Fx(t) + \cancel{Gu(t)} + v_1(t) & \text{state equation}\\
        y(t) = Hx(t) + v_2(t) & \text{output equation}
    \end{cases}
\]
\[
    x(t) = \begin{bmatrix}
        x_1(t) \\
        x_2(t) \\
        \vdots \\
        x_n(t)
    \end{bmatrix}
    \qquad
    \left(\;u(t) = \begin{bmatrix}
        u_1(t) \\
        u_2(t) \\
        \vdots \\
        u_m(t)
    \end{bmatrix}\;\right)
    \qquad
    y(t) = \begin{bmatrix}
        y_1(t) \\
        y_2(t) \\
        \vdots \\
        y_p(t)
    \end{bmatrix}
\]

System with $n$ states, ($m$ inputs) and $p$ outputs.

$v_1(t)$ is a vector white-noise.
\[
    v_1(t) \sim WN(0, V_1) \qquad v_1(t) = \begin{bmatrix}
        v_{11}(t) \\
        v_{12}(t) \\
        \vdots \\
        v_{1n}(t)
    \end{bmatrix}
\]

$v_1(t)$ is called state noise or model noise.
It accounts for the modelling errors of the state equation of the system.

\begin{itemize}
    \item $E[v_1(t)] = \vec{0}$
    \item $E[v_1(t) \cdot v_1(t)^T] = V_1$, where $V_1$ is an $n\times n$ covariance matrix (symmetric and semi-definite positive)
    \item $E[v_1(t) \cdot v_1(t-\tau)^T] = 0 \quad \forall t \forall \tau \ne 0$
\end{itemize}

$v_2(t)$ is a vector white-noise.
\[
    v_2(t) \sim WN(0, V_2) \qquad v_2(t) = \begin{bmatrix}
        v_{21}(t) \\
        v_{22}(t) \\
        \vdots \\
        v_{2p}(t)
    \end{bmatrix}
\]

$v_2(t)$ is called output noise or measurement (or sensor) noise.
\begin{itemize}
    \item $E[v_2(t)] = \vec{0}$
    \item $E[v_2(t) \cdot v_2(t)^T] = V_2$, where $V_2$ is a $p\times p$ covariance matrix (symmetric and \textbf{definite} positive)
    \item $E[v_2(t) \cdot v_2(t-\tau)^T] = 0 \quad \forall t \forall \tau \ne 0$
\end{itemize}

Assumptions of the relationships between $v_1(t)$ and $v_2(t)$:
\[
    E[v_1(t) \cdot v_2(t-\tau)^T] = \underbrace{V_{12}}_{n\times p} = \begin{cases}
        0 & \text{if } \tau \ne 0 \\
        \text{can be non-zero} & \text{if } \tau = 0
    \end{cases}
\]

They can be correlated but only at the same time (in practice $V_{12}=0$ is the most common assumption).

Since the system $S$ is dynamic we need to define the initial conditions:
\[
    E[x(1)] = \underbrace{X_0}_{n\times 1} \qquad E[\left(x(1) - x(0)\right)\left(x(1)-x(0)\right)^T] = \underbrace{P_0}_{n\times n} \ge 0
\]

If $P_0 = 0$ the initial state is perfectly known.

Finally we assume that the two noises $v_1(t)$ and $v_2(t)$ are uncorrelated with the initial state:
\[
    x(1) \perp v_1(t) \qquad x(1) \perp v_2(t)
\]

\subsection{Basic Solution}

\begin{align*}
    & \hat{x}(t+1|t) = F\hat{x}(t|t-1) + K(t)e(t) && \qquad \text{state equation} \\
    & \hat{y}(t|t-1) = H\hat{x}(t|t-1) &&\qquad \text{output equation} \\
    & e(t) = y(t) - \hat{y}(t|t-1) &&\qquad \text{output prediction error} \\
    & K(t) = \left( FP(t)H^T+V_{12} \right) \left( HP(t)H^T+V_2 \right)^{-1} &&\qquad \text{gain of the K.F.} \\
    & P(t+1) = \left( FP(t)F^T + V_1 \right) + &&\\
    & - \left( FP(t)H^T + V_{12} \right)\left( HP(t)H^T + V_{2} \right)^{-1}\left( FP(t)H^T + V_{12} \right)^T && \qquad\text{difference Riccati equation}
\end{align*}

These equations must be completed with 2 initial conditions (2 are dynamic equations)

\begin{align*}
    \hat{x}(1|0) = E[x(1)] = X_0 & \text{state equation} \\
    P(1) = \text{var}[x(1)] = P_0 & \text{DRE}
\end{align*}

\begin{remark}[Structure or $K(t)$ and D.R.E]
    Notice that $K(t)$ and DRE have  a\emph{blockset} structure having this form: $AP(t)B^T+N$

    There are 3 different types of blocks:
    \begin{align*}
        \text{state} \qquad& FP(t)F^T+V_1 \\
        \text{output} \qquad& HP(t)H^T+V_2 \\
        \text{mix} \qquad& FP(t)H^T+V_{12}
    \end{align*}

    \begin{align*}
        \text{gain} \qquad& (\text{mix})(\text{output})^{-1} \\
        \text{DRE} \qquad& (\text{state}) - (\text{mix})(\text{output})^{-1}(\text{mix})^T
    \end{align*}
\end{remark}

\begin{remark}[Riccati equation]
    Riccati equation is a special type of nonlinear matrix difference equation.

    Notice that DRE is an autonomous, non-linear, discrete time, multi-variable system, described by a non-linear difference matrix equation.
    \[
        \text{DRE: } P(t+1) = f(P(t)) \qquad P(1) = P_0
    \]
\end{remark}

\begin{remark}[Existance of DRE]
    In order to guarantee the existance of DRE for all $t$ the only critical part is the inversion of the \emph{output} block:
    \[
        ( \underbrace{HP(t)H^T}_{\ge 0} + \underbrace{V_2}_{>0})^{-1} \qquad \text{Thanks to $V_2>0$ it's invertible}
    \]
\end{remark}

\begin{remark}[Meaning of $P(t)$]
    It can be proven that $P(t)$ has a very important meaning.

    \[
        P(t) = \text{var}[x(t) - \hat{x}(t|t-1)] = E[(x(t) - \hat{x}(t|t-1))(x(t) - \hat{x}(t|t-1))^T]
    \]

    $P(t)$ is the covariance of the 1-step prediction error of the state.
\end{remark}
