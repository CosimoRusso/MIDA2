\begin{remark}[Strictly proper systems]
    Notice that for strictly proper systems the delay $k \ge 1$
\end{remark}
\missingfigure{Graphs about strictly proper systems}

\subsection{Representation \#3: convolution of the input with the Impulse Response (IR)}
The third way to represent a system is through the convolution of the input with the \emph{Impulse Response (IR)}.
\missingfigure{Graphs about convolution}
It can be proven that the input-output relationship from $u(t)$ to $y(t)$ can be written as
\[ y(t) = \omega(0) u(t) + \omega(1) u(t-1) + \omega(2) u(t-2) + \cdots \]
It can be rewritten as follows
\[ y(t) = \sum_{k=0}^{\infty} \omega(k) u(t-k) \]
From this, it is clear that $y(t)$ is the convolution of IR with the input signal.
\missingfigure{Block diagram of system}

\section{Transformation between representations}
\missingfigure{Figure of translation between representations}
\subsection{State Space to Transfer function}
Consider a strictly proper system with the following state space representation:
\[
\begin{cases}
    x(t+1) = F x(t) + G u(t)\\
    y(t) = H x(t) + \cancelto{0}{D u(t)}\\
\end{cases}
\Rightarrow
\begin{cases}
    x(t+1) = F x(t) + G u(t)\\
    y(t) = H x(t)\\
\end{cases}
\]
From the system we get
\[ z x(t) = F x(t) + G u(t) \Rightarrow x(t) = (zI - F)^{-1} G u(t) \]
\[ \Rightarrow y(t) = H (zI - F)^{-1} G \cdot u(t) \]
Thus, the transfer function is
\[ W(z) = H(zI - F) ^ {-1} G \]

\todo{example env}
Example:
\begin{align*}
    F = \begin{bmatrix}
        1 & 0\\
        \frac{1}{2} & 2\\
    \end{bmatrix}
    &&
    G = \begin{bmatrix}
        1\\
        1\\
    \end{bmatrix}
    &&
    H = \begin{bmatrix}
        1 & 0\\
    \end{bmatrix}
    &&
    D = 0
\end{align*}
\begin{align*}
W(z) &=
\begin{bmatrix}
    1 & 0\\
\end{bmatrix}
\left( \begin{bmatrix}
    z & 0\\
    0 & z\\
\end{bmatrix}
-
\begin{bmatrix}
    1 & 0 \\
    \frac{1}{2} & 2\\
\end{bmatrix}\right)^{-1}
\begin{bmatrix}
    1\\
    1\\
\end{bmatrix}
= \begin{bmatrix}
    1 & 0\\
\end{bmatrix}
\begin{bmatrix}
    z-1 & 0\\
    -\frac{1}{2} & z-2\\
\end{bmatrix}^{-1}
\begin{bmatrix}
    1\\
    1\\
\end{bmatrix}\\
&= \begin{bmatrix}
    1 & 0\\
\end{bmatrix}
\frac{1}{(z-1)(z-2)}
\begin{bmatrix}
    z-2 & 0\\
    \frac{1}{2} & z-1\\
\end{bmatrix}
\begin{bmatrix}
    1\\
    1\\
\end{bmatrix}
=
\frac{1}{(z-1)(z-2)}
\begin{bmatrix}
    z-2 & 0\\
\end{bmatrix}
\begin{bmatrix}
    1\\
    1\\
\end{bmatrix}\\
&=
\frac{\cancel{z-2}}{(z-1)\cancel{(z-2)}} = \frac{1}{z-1} = \frac{1}{1-z^{-1}} z^{-1}
\end{align*}
Notice that in this case we only have one pole, but the system is of order two; this comes from the fact that part of the system is non observable.

\subsection{Transfer function to State Space}
This conversion is not very used in practice and it is called the \emph{realization} of a transfer function into a state space model.

Note that the state space representation is not unique, so from a single transfer function we can get infinite different equivalent state space models.

\todo{example env}
Example: Control realization

We assume that the system is strictly proper and that the denominator is monic.
\[ W(z) = \frac{b_0 z^{n-1} + b_1 z^{n-2} + \dots + b_{n-1}}{z^n + a_1 z^{n-1} + a_2 z^{n-2} + \dots + a_n} \]

The formula for the control realization of $W(z)$ is
\begin{align*}
    F = \begin{bmatrix}
        0 & 1 & 0 & \cdots & 0\\
        0& 0 & 1 & \ddots & \vdots \\
        \vdots & \vdots & \ddots & \ddots & 0\\
        0 & 0 & \cdots & 0 & 1\\
        -a_n & -a_{n-1} & \multicolumn{2}{c}{\cdots} & -a_1\\
    \end{bmatrix}
    &&
    G = \begin{bmatrix}
        0\\
        0\\
        0\\
        \vdots\\
        1\\
    \end{bmatrix}
    &&
    H = \begin{bmatrix}
        b_{n-1} & b_{n-2} & \cdots & b_0\\
    \end{bmatrix}
    &&
    D = 0
\end{align*}