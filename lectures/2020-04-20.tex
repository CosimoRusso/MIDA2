\begin{remark}[Strictly proper systems]
    Notice that for strictly proper systems the delay $k \ge 1$
\end{remark}
\missingfigure{Graphs about strictly proper systems}

\subsection{Representation \#3: convolution of the input with the Impulse Response (IR)}
The third way to represent a system is through the convolution of the input with the \emph{Impulse Response (IR)}.
\missingfigure{Graphs about convolution}
It can be proven that the input-output relationship from $u(t)$ to $y(t)$ can be written as
\[ y(t) = \omega(0) u(t) + \omega(1) u(t-1) + \omega(2) u(t-2) + \cdots \]
It can be rewritten as follows
\[ y(t) = \sum_{k=0}^{\infty} \omega(k) u(t-k) \]
From this, it is clear that $y(t)$ is the convolution of IR with the input signal.
\missingfigure{Block diagram of system}

\section{Transformation between representations}
\missingfigure{Figure of translation between representations}
\subsection{State Space to Transfer Function}
Consider a strictly proper system with the following state space representation:
\[
\begin{cases}
    x(t+1) = F x(t) + G u(t)\\
    y(t) = H x(t) + \cancelto{0}{D u(t)}\\
\end{cases}
\Rightarrow
\begin{cases}
    x(t+1) = F x(t) + G u(t)\\
    y(t) = H x(t)\\
\end{cases}
\]
From the system we get
\[ z x(t) = F x(t) + G u(t) \Rightarrow x(t) = (zI - F)^{-1} G u(t) \]
\[ \Rightarrow y(t) = H (zI - F)^{-1} G \cdot u(t) \]
Thus, the transfer function is
\[ W(z) = H(zI - F) ^ {-1} G \]

\begin{example}
\begin{align*}
    F = \begin{bmatrix}
        1 & 0\\
        \frac{1}{2} & 2\\
    \end{bmatrix}
    &&
    G = \begin{bmatrix}
        1\\
        1\\
    \end{bmatrix}
    &&
    H = \begin{bmatrix}
        1 & 0\\
    \end{bmatrix}
    &&
    D = 0
\end{align*}
\begin{align*}
W(z) &=
\begin{bmatrix}
    1 & 0\\
\end{bmatrix}
\left( \begin{bmatrix}
    z & 0\\
    0 & z\\
\end{bmatrix}
-
\begin{bmatrix}
    1 & 0 \\
    \frac{1}{2} & 2\\
\end{bmatrix}\right)^{-1}
\begin{bmatrix}
    1\\
    1\\
\end{bmatrix}
= \begin{bmatrix}
    1 & 0\\
\end{bmatrix}
\begin{bmatrix}
    z-1 & 0\\
    -\frac{1}{2} & z-2\\
\end{bmatrix}^{-1}
\begin{bmatrix}
    1\\
    1\\
\end{bmatrix}\\
&= \begin{bmatrix}
    1 & 0\\
\end{bmatrix}
\frac{1}{(z-1)(z-2)}
\begin{bmatrix}
    z-2 & 0\\
    \frac{1}{2} & z-1\\
\end{bmatrix}
\begin{bmatrix}
    1\\
    1\\
\end{bmatrix}
=
\frac{1}{(z-1)(z-2)}
\begin{bmatrix}
    z-2 & 0\\
\end{bmatrix}
\begin{bmatrix}
    1\\
    1\\
\end{bmatrix}\\
&=
\frac{\cancel{z-2}}{(z-1)\cancel{(z-2)}} = \frac{1}{z-1} = \frac{1}{1-z^{-1}} z^{-1}
\end{align*}
Notice that in this case we only have one pole, but the system is of order two; this comes from the fact that part of the system is non observable.

\end{example}

\subsection{Transfer Function to State Space}
This conversion is not very used in practice and it is called the \emph{realization} of a transfer function into a state space model.

Note that the state space representation is not unique, so from a single transfer function we can get infinite different equivalent state space models.

\subsubsection{Control realization}

We assume that the system is strictly proper and that the denominator is monic.
\[ W(z) = \frac{b_0 z^{n-1} + b_1 z^{n-2} + \dots + b_{n-1}}{z^n + a_1 z^{n-1} + a_2 z^{n-2} + \dots + a_n} \]

The formula for the control realization of $W(z)$ is
\begin{align*}
    F = \begin{bmatrix}
        0 & 1 & 0 & \cdots & 0\\
        0& 0 & 1 & \ddots & \vdots \\
        \vdots & \vdots & \ddots & \ddots & 0\\
        0 & 0 & \cdots & 0 & 1\\
        -a_n & -a_{n-1} & \multicolumn{2}{c}{\cdots} & -a_1\\
    \end{bmatrix}
    &&
    G = \begin{bmatrix}
        0\\
        0\\
        0\\
        \vdots\\
        1\\
    \end{bmatrix}
    &&
    H = \begin{bmatrix}
        b_{n-1} & b_{n-2} & \cdots & b_0\\
    \end{bmatrix}
    &&
    D = 0
\end{align*}
\begin{example}
    Consider the transfer function $W(z)$
    \[ W(z) = \frac{2z^2 + \frac{1}{2}z + \frac{1}{4}}{z^3 + \frac{1}{4}z^2 + \frac{1}{3}z + \frac{1}{5}} \]
    The control realization is
    \begin{align*}
        F = \begin{bmatrix}
            0 & 1 & 0\\
            0 & 0 & 1\\
            -\frac{1}{5} & -\frac{1}{3} & -\frac{1 }{4}\\
        \end{bmatrix}
        &&
        G = \begin{bmatrix}
            0\\
            0\\
            1\\
        \end{bmatrix}
        &&
        H = \begin{bmatrix}
            \frac{1}{4} & \frac{1}{2} & 2\\
        \end{bmatrix}
        &&
        D = 0
    \end{align*}
\end{example}

\subsection{Transfer Function to Impulse Response}
To get the IR from a transfer function $W(z)$ is sufficient to make the $\infty$-long division between the numerator and denominator of $W(z)$
\begin{example}
    Consider the transfer function
    \[ W(z) = \frac{1}{z-\frac{1}{2}} = \frac{z^{-1}}{1-\frac{1}{2}z^{-1}}
        = 0 z^{-0} + 1 z^{-1} + \frac{1}{2}z^{-2} + \frac{1}{4}z^{-3} + \cdots \]
    Thus the IR is $\omega(0) = 0$, $\omega(1) = 1$, $\omega(2) = \frac{1}{2}$, $\omega(3) = \frac{1}{4}$, $\dots$

    In this case there is also a quicker way
    \[ y(t) = \frac{z^{-1}}{1-\frac{1}{2}z^{-1}} u(t) = \left( z^{-1} \frac{1}{1-\frac{1}{2}z^{-1}} \right) u(t) \]
    Remembering that for geometric series we have \[ \sum_{k = 0}^{\infty} a^k = \frac{1}{1-a} \text{ if } |a| < 1 \]
    We can rewrite $y(t)$ as follows
    \[ y(t) = \left( z^{-1} \sum_{k=0}^{\infty} \left( \frac{1}{2} z^{-1} \right)^{k} \right) u(t) = \left( 0 + 1 z^{-1} + \frac{1}{2}z^{-2} + \frac{1}{4}z^{-3} + \cdots \right) u(t) \]
\end{example}

\subsection{Impulse Response to Transfer Function}
\begin{definition}
    Given a discrete-time signal $s(t)$ such that $\forall t < 0: s(t) = 0$, its \emph{Z-Transform} is defined as
    \[ \mathcal{Z} \left( s(t) \right) = \sum_{t = 0}^{\infty} s(t) z^{-t} \]
\end{definition}
Given this, it can be proven that
\[ W(z) = \mathcal{Z}\left( \omega(t) \right) = \sum_{t = 0}^{\infty} \omega(t) z^{-t} \]
This means that the transfer function of a system is the $\mathcal{Z}$-transform of a special signal, that is the impulse response of the system.

\begin{remark}
    This formula cannot be used in practice to transform an IR representation to a TF representation.
    This is because we need infinite points of the impulse response, and the impulse response must be noise-free.
    Thus, this transformation is only theoretical.
\end{remark}

\subsection{State Space to Impulse Reponse}
Consider the following state space model, with initial conditions $x(0) = 0$ and $y(0) = 0$
\[
    \begin{cases}
        x(t+1) = F x(t) + G u(t)\\
        y(t) = H x(t)\\
    \end{cases}
\]
We have that
\begin{align*}
    x(1) &= \cancelto{0}{F x(0)} + G u(0) = G u(0)\\
    y(1) &= H x(1) = H G u(0)\\
\end{align*}
\begin{align*}
    x(2) &= F x(1) + G u(1) = F G u(0) + G u(1)\\
    y(2) &= H x(2) = H F G u(0) + H G u(1)\\
\end{align*}
\begin{align*}
    x(3) &= F x(2) + G u(2) = F^2 G u(0) + F G u(1) + G u(2)\\
    y(3) &= H x(3) = H F^2 G u(0) + H F G u(1) + H G u(2)\\
\end{align*}
This can be generalized to
\[ y(t) = 0 u(t) + H G u(t-1) + H F G u(t-2) + H F^2 G u(t-3) + \cdots \]
Thus, the impulse response is
\[
    \omega(t) =
    \begin{cases}
        0 \text{ if } t = 0\\
        H F^{t-1} G \text{ if } t > 0\\
    \end{cases}
\]

\subsection{Summary of transformations}
Notice that the IR representation is very easy to obtain experimentally, since we only need to measure the system response to the impulse signal.
However, given the IR representation, it is difficult to get to the other representations, since the transformation between from IR to TF is only theoretical.
Moving from the IR to the SS representation is the key task of the \emph{Subspace-based State Space System Identification}, also known as \emph{4SID methods}.

\section{Subspace-based State Space System Identification (4SID)}
The original 4SID method starts from the measurement of the system output in a very simple experiment, that is the \emph{impulse experiment}.
\missingfigure{Impulse response}
The fundamental idea is that it is very simple to make this experiment, that is to measure the output of system given an impulse signal as the input.
The real problem is to identify a model $\left\{ \hat{F}, \hat{G}, \hat{H} \right\}$ starting from $\left\{ \omega(0), \omega(1), \omega(2), \cdots \right\}$.

We will see the solution of this problem in two steps:
\begin{enumerate}
    \item IR measurement is assumed to be noise-free.
    This is easier with respect to the original problem, but it is also not realistic.
    \item IR is measured with noise
    \[ \widetilde{\omega}(t) = \omega(t) + \eta(t) \quad t = 0, 1,\dots, N \]
    \begin{itemize}
        \item $\widetilde{\omega}(t)$: measured noisy IR
        \item $\omega(t)$: "true" noise-free IR
        \item $\eta(t)$: measurement noise (e.g. WN)
    \end{itemize}
\end{enumerate}
\begin{remark}
    We will see in detail only 4SID when the experiment is an impulse-experiment, which is the first and original version of 4SID.
    However 4SID can be extended to any generic input signal $\left\{ u(1), u(2), \cdots, u(N) \right\}$ that is sufficiently exciting.
\end{remark}
\begin{remark}[Unstable system]
    In case of an unstable system the measurements must be collected in a closed-loop experiment.
    Indeed, if the experiment was open-loop, the experiment would be unfeasible.
    \missingfigure{Unstable system}
    \missingfigure{Closed loop system}
\end{remark}