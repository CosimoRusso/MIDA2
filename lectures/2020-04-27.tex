\newlecture{Sergio Savaresi}{27/04/2020}

The general ideal of the method is:
\begin{itemize}
    \item Make a set of ``single sinusoid'' (``single-tune'') excitation experiments
    \item From each experiment estimate a single point of the frequency response of the system
    \item Fit the estimated and modeled frequency response to obtain the optimal model
\end{itemize}

\begin{example}
    \missingfigure{car drawing}

    This is the dynamic relationship linking the input (steer) to the output (yaw-rate).
    This kind of relationship is very important for stability control design and autonomous car.

    There are 3 possible situations:
    \begin{figure}[H]
        \centering
        \begin{tikzpicture}[node distance=1.5cm,auto,>=latex']
            \node[int] (sys1) {1};
            \node[left of=sys1] (in1) {$\delta_F$};
            \node[right of=sys1] (out1) {$\omega_z$};
            \draw[->, red] (in1) -- (sys1);
            \draw[->] (sys1) -- (out1);

            \node[int, right of=sys1, node distance=4cm] (sys2) {2};
            \node[left of=sys2] (in2) {$\delta_R$};
            \node[above of=sys2, node distance=1cm] (in2b) {$\delta_F$};
            \node[right of=sys2] (out2) {$\omega_z$};
            \draw[->, red] (in2) -- (sys2);
            \draw[->] (in2b) -- (sys2);
            \draw[->] (sys2) -- (out2);

            \node[int, right of=sys2, node distance=4cm] (sys3) {3};
            \node[left of=sys3, yshift=0.3cm] (in3a) {$\delta_F$};
            \node[left of=sys3, yshift=-0.3cm] (in3b) {$\delta_R$};
            \node[right of=sys3] (out3) {$\omega_z$};
            \draw[->, red] (in3a) -- (sys3);
            \draw[->, red] (in3b) -- (sys3);
            \draw[->] (sys3) -- (out3);
        \end{tikzpicture}
    \end{figure}

    \begin{enumerate}
        \item The control variable is the front steer: application is autonomous car
        \item The driver controls $\delta_F$ which is a measurable disturbance, the system controls the rear steer
        \item Both $\delta_R$ and $\delta_F$ are control variables: application high performance autonomous car
    \end{enumerate}
\end{example}

\paragraph{Step 1} In the experiment design step we first have to select a set of excitation frequencies.

\begin{figure}[H]
    \centering
    \begin{tikzpicture}[node distance=1.5cm,auto,>=latex']
        \draw[->] (0,0) -- (5,0) node[right] {$\omega$};
        \draw (0,0.1) -- (0,-0.1) node[below] {$0$};
        \draw (0.8,0.1) -- (0.8,-0.1) node[below] {$\omega_1$};
        \draw (1.6,0.1) -- (1.6,-0.1) node[below] {$\omega_2$};
        \draw (2.4,0.1) -- (2.4,-0.1) node[below] {$\omega_3$};
        \draw (3.2,0.1) -- (3.2,-0.1) node[below] {$\ldots$};
        \draw (4,0.1) -- (4,-0.1) node[below] {$\omega_H$};
    \end{tikzpicture}
\end{figure}

The maximum frequency is $\omega_H$.
We have $\{\omega_1, \omega_2, \cdots, \omega_H\}$ usually evenly spaced ($\Delta \omega$ is constant).
$\omega_H$ must be selected according to the bandwidth of the control system.

We make $H$ independent experiments.

\paragraph{Experiment \#1} $\omega_1$
\begin{figure}[H]
    \centering
    \begin{tikzpicture}[node distance=1.5cm,auto,>=latex']
        \node[int] (sys) {system};
        \node[left of=sys, node distance=3cm] (in) {};
        \node[left of=in] (in2) {};
        \node[right of=sys, node distance=2.5cm] (out) {};
        \node[right of=out, node distance=2cm] (out2) {};
        \node[below of=in, node distance=0.5cm] {$u_1(t) = A_1\sin(\omega_1t)$};
        \draw[xshift=-4cm,yshift=0.5cm,scale=0.2,domain=0:10,smooth,variable=\x] plot ({\x},{sin(\x*180/3.14)});

        \node[below of=out, node distance=0.5cm] {$y_1(t)$};
        \draw[xshift=1.5cm,yshift=0.5cm,scale=0.2,domain=0:10,smooth,variable=\x] plot ({\x},{1.5*sin(\x*180/3.14+30)});
        \draw[->] (in2) -- (sys);
        \draw[->] (sys) -- (out2);
    \end{tikzpicture}
\end{figure}

\paragraph{Experiment \#2} $\omega_2$
\begin{figure}[H]
    \centering
    \begin{tikzpicture}[node distance=1.5cm,auto,>=latex']
        \node[int] (sys) {system};
        \node[left of=sys, node distance=3cm] (in) {};
        \node[left of=in] (in2) {};
        \node[right of=sys, node distance=2.5cm] (out) {};
        \node[right of=out, node distance=2cm] (out2) {};
        \node[below of=in, node distance=0.5cm] {$u_2(t) = A_2\sin(\omega_2t)$};
        \draw[xshift=-4cm,yshift=0.5cm,scale=0.2,domain=0:10,smooth,variable=\x] plot ({\x},{sin(2*\x*180/3.14)});

        \node[below of=out, node distance=0.5cm] {$y_2(t)$};
        \draw[xshift=1.5cm,yshift=0.5cm,scale=0.2,domain=0:10,smooth,variable=\x] plot ({\x},{1.5*sin(2*\x*180/3.14+30)});
        \draw[->] (in2) -- (sys);
        \draw[->] (sys) -- (out2);
    \end{tikzpicture}
\end{figure}

\paragraph{Experiment \#$H$} $\omega_H$
\begin{figure}[H]
    \centering
    \begin{tikzpicture}[node distance=1.5cm,auto,>=latex']
        \node[int] (sys) {system};
        \node[left of=sys, node distance=3cm] (in) {};
        \node[left of=in] (in2) {};
        \node[right of=sys, node distance=2.5cm] (out) {};
        \node[right of=out, node distance=2cm] (out2) {};
        \node[below of=in, node distance=0.5cm] {$u_H(t) = A_H\sin(\omega_Ht)$};
        \draw[xshift=-4cm,yshift=0.5cm,scale=0.2,domain=0:10,smooth,variable=\x,samples=50] plot ({\x},{sin(5*\x*180/3.14)});

        \node[below of=out, node distance=0.5cm] {$y_H(t)$};
        \draw[xshift=1.5cm,yshift=0.5cm,scale=0.2,domain=0:10,smooth,variable=\x,samples=50] plot ({\x},{1.5*sin(5*\x*180/3.14+30)});
        \draw[->] (in2) -- (sys);
        \draw[->] (sys) -- (out2);
    \end{tikzpicture}
\end{figure}


\begin{remark}
    The amplitudes $A_1$, $A_2$, \ldots, $A_H$ can be equal (constant) or, more frequently, they decrease as the frequency increases to fulfill the power constraint on the input.

    \paragraph{Example} $\delta(t)$ is the steering angle (moved by an actuator).
    The requested steer torque is proportional to $\delta$: $T(t) = K \delta(t)$.
    Therefore the steer power is proportional to $T(t) \dot{\delta}(t) = K \delta(t)\dot{\delta}(t)$.
    If $\delta(t) = A_i\sin(\omega_it)$ the steering power is $KA_i\sin(\omega_it)\omega_iA_i\cos(\omega_it)$ which is proportional to $KA_i^2\omega_i$.

    If we have a limit to this power, this power should be constant during the $H$ experiments.
    \[KA_i^2\omega_i = \text{const} \qquad A_i=\sqrt{\frac{\text{const}}{K\omega_i}}\]
\end{remark}

Focusing on the $i$-th experiment.
\begin{figure}[H]
    \centering
    \begin{tikzpicture}[node distance=1.5cm,auto,>=latex']
        \node[int] (sys) {system};
        \node[left of=sys, node distance=3cm] (in) {};
        \node[left of=in] (in2) {};
        \node[right of=sys, node distance=2.5cm] (out) {};
        \node[right of=out, node distance=2cm] (out2) {};
        \node[below of=in, node distance=0.5cm] {$u_i(t) = A_i\sin(\omega_it)$};
        \draw[xshift=-4cm,yshift=0.5cm,scale=0.2,domain=0:10,smooth,variable=\x] plot ({\x},{sin(2*\x*180/3.14)});

        \node[below of=out, node distance=0.5cm] {$y_i(t)$};
        \draw[xshift=1.5cm,yshift=0.5cm,scale=0.2,domain=0:10,smooth,variable=\x,samples=100] plot ({\x},{1.5*sin(2*\x*180/3.14+30)+rand/3});
        \draw[->] (in2) -- (sys);
        \draw[->] (sys) -- (out2);
    \end{tikzpicture}
\end{figure}

\begin{remark}
    If the system is LTI (linear time-invariant), the frequency response theorem says the if the input is a sine input of frequency $\omega_i$ the output must be a sine with frequency $\omega_i$.
\end{remark}

However $y_i(t)$ in real applications is not a perfect sinusoid.
\begin{itemize}
    \item Noise on output measurements
    \item Noise on the system (not directly on the output)
    \item (Small) non-linear effects (that we neglect)
\end{itemize}

In pre-processing of I/O data we want to extract from $y_i(t)$ a perfect sinusoid of frequency $\omega_i$.
We force the assumption that the system is LTI, so the output must be a pure sine wave of frequency $\omega_i$ (all the remaining signal is noise).

The model of the output signal is
\[ \hat{y}_i = B_i \sin(\omega_it+\phi_i) = a_i\sin(\omega_it) + b_i\cos(\omega_it) \]
There are 2 unknowns: $B_i$ and $\phi_i$ (or $a_i$ and $b_i$).
We try to estimate $a_i$ and $b_i$ since they are \emph{linear}.

The unknown parameters are $a_i$ and $b_i$ and we can find them by parametric identification.
\[ \{ \hat{a}_i, \hat{b}_i \} = \argmin_{\{a_i, b_i\}} J_N(a_i, b_i) \]
\[ J_N(a_i, b_i) = \frac{1}{N} \sum_{t=1}^N ( \underbrace{y_i(t)}_{\text{measurement}} \underbrace{- a_i\sin(\omega_it) - b_i\cos(\omega_it)}_\text{model output})^2 \]

$J_N$ is a quadratic function of $a_i$ and $b_i$, so we can solve the problem explicitly.

\begin{align*}
    \frac{\partial J_N}{\partial a_i} &= \frac{2}{N} \sum_{t=1}^N (-\sin(\omega_it))(y_i(t) - a_i\sin(\omega_it)-b_i\cos(\omega_it)) = 0 \\
    \frac{\partial J_N}{\partial b_i} &= \frac{2}{N} \sum_{t=1}^N (-\cos(\omega_it))(y_i(t) - a_i\sin(\omega_it)-b_i\cos(\omega_it)) = 0
\end{align*}

\[
    \begin{bmatrix}
        \sum_{t=1}^N \sin(\omega_it)^2 & \sum_{t=1}^N \sin(\omega_it)\cos(\omega_it) \\
        \sum_{t=1}^N \sin(\omega_it)\cos(\omega_it) & \sum_{t=1}^N \cos(\omega_it)^2
    \end{bmatrix}
    \begin{bmatrix}
        a_i \\ b_i
    \end{bmatrix} =
    \begin{bmatrix}
        \sum_{t=1}^N y_i(t)\sin(\omega_it) \\
        \sum_{t=1}^N y_i(t)\cos(\omega_it)
    \end{bmatrix}
\]

At this point we prefer to go back to a \emph{sin-only} form ($B_i$, $\phi_i$):
\[
    \hat{B}_i\sin(\omega_it + \phi_i) = \hat{B}_i\sin(\omega_it)\cos(\hat{\phi}_i) + \hat{B}_i\cos(\omega_it)\sin(\hat{\phi_i}) = \hat{a}_i\sin(\omega_it) + \hat{b}_i\cos(\omega_it)
\]

\begin{align*}
    \hat{B}_i\cos(\hat{\phi}_i) = \hat{a}_i \\
    \hat{B}_i\sin(\hat{\phi}_i) = \hat{b}_i \\
\end{align*}
\[
    \frac{\hat{b}_i}{\hat{a}_i} = \frac{\sin\hat{\phi}_i}{\cos{\hat{\phi}_i}} = \tan(\hat{\phi_i}) \qquad \hat{\phi}_i = \arctan \left( \frac{\hat{b}_i}{\hat{a}_i} \right)
\]
\[
    \hat{B}_i = \frac{\frac{\hat{a}_i}{\cos\hat{\phi}_i} + \frac{\hat{b}_i}{\sin\hat{\phi}_i}}{2}
\]

Repeating the experiment and pre-processing for the $H$ experiments we obtain

\begin{align*}
    \{ \hat{B}_1, \hat{\phi}_1 \} &\implies \frac{\hat{B}_1}{A_1} e^{j\hat{\phi_1}} \\
    \{ \hat{B}_2, \hat{\phi}_2 \} &\implies \frac{\hat{B}_2}{A_2} e^{j\hat{\phi_2}} \\
    \vdots& \\
    \{ \hat{B}_H, \hat{\phi}_H \} &\implies \frac{\hat{B}_H}{A_H} e^{j\hat{\phi_H}} \\
\end{align*}

We obtain $H$ complex numbers, the estimated $H$ points of the frequency response of the transfer function $W(z)$ from the input $u(t)$ to the output $y(t)$ of the system.

\missingfigure{Fig6}

At the end of \emph{step 1} we have a frequency-domain dataset ($H$ values) representing $H$ estimated points of the frequency response of the system.

\paragraph{Step 2} Model class selection (T.F.)

\[
    \mathcal{M}(\theta): W(z; \theta) = \frac{b_0+b_1z^{-1}+\cdots+b_pz^{-p}}{1+a_1z^{-1}+\cdots+a_nz^{-n}}z^{-1}
    \qquad
    \theta = \begin{bmatrix}
        a_1 \\ \vdots \\ a_n \\ b_0 \\ \vdots \\ b_p
    \end{bmatrix}
\]

\begin{remark}[Model order selection]
    In this case the order is composed by 2 parameters $n$ and $p$.
    Use cross-validation approach (or visual fitting in the Bode diagram).
\end{remark}

\paragraph{Step 3} New performance index (frequency domain).

\[
    J_H(\theta) = \frac{1}{H} \sum_{i=1}^H \left(W(e^{j\omega_i}; \theta) - \frac{\hat{B}_i}{A_i}e^{j\hat{\phi}_i} \right)^2
\]

\paragraph{Step 4} Optimization

\[
    \hat{\theta} = \argmin_\theta J_H(\theta)
\]

Usually $J_H(\theta)$ is a non-quadratic and non-convex function, iterative optimization methods are needed.

\begin{remark}[Frequency bandwidth selection $\omega_H =\; ?$]
    Theoretically the standard best solution should be $H$ points distributed uniformly from 0 to $\Omega_N$ (Nyquist).

    In practice it's better to concentrate the experimental effort in a smaller and more focused bandwidth.

    \begin{figure}[H]
        \centering
        \begin{tikzpicture}[node distance=1.5cm,auto,>=latex']
            \draw[->] (0,0) -- (5,0) node[right] {$\omega$};
            \draw (0,0.1) -- (0,-0.1) node[below] {$0$};
            \draw (0.8,0.1) -- (0.8,-0.1) node[below] {$\omega_c$};
            \draw (2,0.1) -- (2,-0.1) node[below] {$\Omega_N$};
            \draw (4,0.1) -- (4,-0.1) node[below] {$\Omega_S$};
        \end{tikzpicture}
    \end{figure}

    $\omega_c$ is the expected cut-off frequency of the closed system.
    A rule of thumb: $\omega_H \approx 3\omega_c$

    \paragraph{Example} The ESC (electronic stability control) has an expected bandwidth of $\omega_c \approx 2 \text{Hz}$, so $\omega_H \approx 6\text{Hz}$.
\end{remark}
