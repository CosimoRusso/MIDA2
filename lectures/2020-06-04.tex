\chapter{Appendix: Discretization of an analog system}
\newlecture{Sergio Savaresi}{04/06/2020}

\begin{figure}[H]
    \centering
    \begin{tikzpicture}[node distance=2cm,auto,>=latex']
        \node[int,ellipse,align=center] at (0,0) (controller) {Control\\algorithm\\(discrete time)};
        \node[int,align=center] at (6,0) (system) {Physical\\system\\(analog)};
        \node[int] at (3.5,0) (DAC) {D/A};
        \node[int] at (3.5,-2) (ADC) {A/D};

        \draw[->] (controller) -- (DAC);
        \draw[->] (DAC) -- (system) node[pos=0.5] {$\scriptstyle u(t)$};
        \draw[->] (system) -- ++(2,0) node[above] {$y(t)$};
        \draw[->] (7.3,0) |- (ADC);
        \draw[->] (ADC) -| (controller);
    \end{tikzpicture}
\end{figure}

\subsection*{A to D converter}

\begin{figure}[H]
    \centering
    \begin{tikzpicture}[node distance=2cm,auto,>=latex',declare function={
            f(\x) = sin(180*\x/3.14)+1.2;
        }]
        \draw[->] (0,0) -- (5,0) node[below] {$t$};
        \draw[->] (0,0) -- (0,3) node[left] {$y(t)$};

        \draw[domain=0:5,variable=\x,smooth] plot ({\x},{f(\x)});
        \foreach \x in {0,...,9} {
            \draw[dotted] (\x/2,0) -- (\x/2,3);
            \draw[fill] (\x/2,{round(2*f(\x/2))/2}) circle (0.03);
        }
        \foreach \y in {0,...,5} {
            \draw[dotted] (0,\y/2) -- (5,\y/2);
        }

        \draw[<->] (0.5,-0.1) -- (1,-0.1) node[pos=0.5,below] {$\scriptstyle\Delta T$};
        \node[right] at (1.05,-0.33) {\footnotesize (sampling time)};
        \draw[<->] (-0.1,1.5) -- (-0.1,2);
        \node[left,align=right] at (-0.1,1.75) {\footnotesize amplitude\\\footnotesize discretization};
    \end{tikzpicture}
\end{figure}

\begin{description}
    \item[Time discretization] $\Delta T$ is the sampling time
    \item[Amplitude discretization] Number of levels used for discretization, e.g. \emph{10-bit discretization} uses $2^{10}$ levels of amplitude
\end{description}

An high quality of A/D converter:
\begin{itemize}
    \item Can use small $\Delta T$
    \item High number of levels (16-bits)
\end{itemize}


\subsection*{D to A converter}

\begin{figure}[H]
    \centering
    \begin{tikzpicture}[node distance=2cm,auto,>=latex',declare function={
            f(\x) = sin(180*\x/3.14)+1.2;
        }]
        \draw[->] (0,0) -- (5,0) node[below] {$t$};
        \draw[->] (0,0) -- (0,3) node[left] {$y(t)$};

        \draw[domain=0:5,variable=\x,smooth] plot ({\x},{f(\x)});
        \foreach \x in {0,...,9} {
            \draw[dotted] (\x/2,0) -- (\x/2,3);
            \draw[fill] (\x/2,{round(2*f(\x/2))/2}) circle (0.03);
        }
        \foreach \x in {0,...,8} {
            \draw[blue] (\x/2,{round(2*f(\x/2))/2}) -- (\x/2+0.5,{round(2*f(\x/2))/2});
            \draw[blue] (\x/2+0.5,{round(2*f(\x/2))/2}) -- (\x/2+0.5,{round(2*f(\x/2+0.5))/2});
        }
        \foreach \y in {0,...,5} {
            \draw[dotted] (0,\y/2) -- (5,\y/2);
        }
    \end{tikzpicture}
\end{figure}

If $\Delta T$ is sufficiently small, the step-wise analog signal is very similar to a smooth analog signal.

\begin{figure}[H]
    \centering
    \begin{tikzpicture}[node distance=2cm,auto,>=latex',declare function={
            f(\x) = sin(180*\x/3.14)+1.2;
        }]
        \draw[->] (0,0) -- (5,0) node[below] {$t$};
        \draw[->] (0,0) -- (0,3) node[left] {$y(t)$};

        \foreach \x in {0,...,49} {
            \draw[fill] (\x/10,{round(20*f(\x/10))/20}) circle (0.01);
        }
        \foreach \x in {0,...,49} {
            \draw[blue] (\x/10,{round(20*f(\x/10))/20}) -- (\x/10+0.1,{round(20*f(\x/10))/20});
            \draw[blue] (\x/10+0.1,{round(20*f(\x/10))/20}) -- (\x/10+0.1,{round(20*f(\x/10+0.1))/20});
        }
    \end{tikzpicture}
\end{figure}

What is the model of the Digital Perspective?

\begin{figure}[H]
    \centering
    \begin{tikzpicture}[node distance=2cm,auto,>=latex']
        \node[int,align=center,minimum height=2cm] at (0,0) (cont) {Digital\\controller};
        \node[int,align=center,minimum height=2cm] at (6,0) (sys) {Analog\\system};
        \node[int] at (3,0.5) (DAC) {D/A};
        \node[int] at (3,-0.5) (ADC) {A/D};

        \draw[->] (cont.east|-DAC) -- (DAC) node[pos=0.2] {$\scriptstyle u(t)$};
        \draw[<-] (cont.east|-ADC) -- (ADC) node[pos=0.2] {$\scriptstyle y(t)$};
        \draw[->] (DAC) -- (DAC-|sys.west);
        \draw[<-] (ADC) -- (ADC-|sys.west);

        \draw[dashed,red] (1.7,-1.5) rectangle (7.5,1.5);
        \node[below] at (4.6,-1.5) {\footnotesize System from a digital perspective};
    \end{tikzpicture}
\end{figure}

\begin{itemize}
    \item We make B.B. system identification from measured data: directly estimate a discrete-time model
    \item We have a physical W.B. model (continuous time), we need to discretize it
\end{itemize}

The most used approach is the State-Space Transformation.

\[
    S: \begin{cases}
        \dot{x} = Ax + bu \\
        y = Cx + (Du)
    \end{cases}
    \qquad
    \text{sampling time $\Delta T$}
    \qquad
    S: \begin{cases}
        x(t+1) = Fx(t) + Gu(t) \\
        y(t) = Hu(t) + (Du(t))
    \end{cases}
\]

Transformation formulas:
\begin{align*}
    F &= e^{A\Delta T} \\
    G &= \int_0^{\Delta T} e^{A\delta}B\, d\delta \\
    H &= C \\
    D &= D
\end{align*}

\begin{remark}
    How the poles of the continuous time system are transformed?

    Can be proved that the eigenvalues (poles) follow the \emph{sampling transformation rule}.
    \[
        z = e^{s\Delta T} \qquad \lambda_F = e^{\lambda_A \Delta T}
    \]

    \begin{figure}[H]
        \centering
        \begin{tikzpicture}[node distance=2cm,auto,>=latex']
            \draw[->] (-2,0) -- (2,0) node[above] {$\operatorname{Re}$};
            \draw[->] (0,-2) -- (0,2) node[left] {$\operatorname{Im}$};

            \draw[->] (4,0) -- (8,0) node[above] {$\operatorname{Re}$};
            \draw[->] (6,-2) -- (6,2) node[left] {$\operatorname{Im}$};

            \draw[dashed, red, line width=0.4mm] (0,-2) -- (0,1.9);
            \fill [pattern=flexible hatch, pattern color=gray!40] (-2,-2) rectangle (0,2);

            \draw[dashed, red, line width=0.4mm] (6,0) circle (1cm);
            \fill [pattern=flexible hatch, pattern color=gray!40] (6,0) circle (1cm);

            \node at (0,2.5) {$s$-domain};
            \node at (6,2.5) {$z$-domain};

            \draw[fill,blue] (0,0) circle (0.05);
            \draw[fill,blue] (7,0) circle (0.05);

            \draw[->] (2,2) -- (4,2) node[pos=0.5] {$z=e^{s\Delta T}$};
        \end{tikzpicture}
    \end{figure}

    How the zeros in of $S$ in continuous time are transformed into zeros in discrete time?

    Unfortunately there is no simple rule like the poles. We can only say:
    \[
        G(s) = \frac{\text{polynomial in $s$ with $h$ zeros}}{\text{polynomial in $s$ with $k$ poles}} \qquad \text{if $G(s)$ is strictly proper: } k > h
    \]
    \[
        G(z) = \frac{\text{polynomial in $z$ with $k-1$ zeros}}{\text{polynomial in $z$ with $k$ poles}} \qquad \text{$G(z)$ with relative degree 1}
    \]

    We have new $k-h-1$ zeros that are generated by the discretization.
    They are called \emph{hidden zeros}.

    Unfortunately these hidden zeros are frequently outside the unit circle, which means that $G(z)$ is not minimum phase even if $G(s)$ is minimum phase.

    We need for instance GMVC to design the control system.
\end{remark}

Another simple discretization technique frequently used is the discretization of time-derivative $\dot{x}$.

\begin{align*}
    \text{\textbf{eulero backward}} &\qquad \dot{x} \approx \frac{x(t)-x(t-1)}{\Delta T} = \frac{x(t)-z^{-1}x(t)}{\Delta T} = \frac{z-1}{z\Delta T} x(t) \\
    \text{\textbf{eulero forward}} &\qquad \dot{x} \approx \frac{x(t+1)-x(t)}{\Delta T} = \frac{zx(t)-x(t)}{\Delta T} = \frac{z-1}{\Delta T} x(t)
\end{align*}

General formula
\[
    \dot{x}(t) = \left[ \frac{z-1}{\Delta T} \frac{1}{\alpha z + (1-\alpha)} \right]x(t) \qquad \text{with } 0 \le \alpha \le 1
\]
\begin{itemize}
    \item if $\alpha = 0$ it's Eulero Forward
    \item if $\alpha = 1$ it's Eulero Backward
    \item if $\alpha = \frac{1}{2}$ it's Tustin method
\end{itemize}

The critical choice is $\Delta T$ (sampling time).
The general intuitive rule is: the smaller $\Delta T$, the better.

\begin{figure}[H]
    \centering
    \begin{tikzpicture}[node distance=2cm,auto,>=latex',declare function={
        a(\x) = (\x-0.5)^2+2;
        b(\x) = -(\x-1)^2+2.125;
        c(\x) = 2*(\x-2.5)^2+0.625;
        d(\x) = -(\x-3)^2+0.794;
        f(\x) = e^(-\x+2.9517);
        z(\x) = (\x < 0.5) * 2 +
                (0.5 <= \x) * (\x < 0.75) * a(\x) +
                (0.75 <= \x) * (\x < 2) * b(\x) +
                (2 <= \x) * (\x < 2.64) * c(\x) +
                (2.64 <= \x) * (\x < 3.35) * d(\x) +
                (3.35 <= \x) * f(\x);
        z2(\x) = z(\x) * (\x < 1) + z(\x) * (\x >= 1) * (\x < 2.5) * (1-(\x-1)^2/1.5/4);
    }]
        \draw[->] (0,0) -- (5.5,0) node[below] {$\omega$};
        \draw[->] (0,0) -- (0,3) node[left] {$|G|$};
        \draw[samples=50,domain=0:5,variable=\x,smooth] plot ({\x},{z(\x)});
        \draw[samples=50,red,domain=0:2.5,variable=\x,smooth] plot ({\x},{z2(\x)});
        \draw[dotted] (2.5,0.625) -- (2.5,0) node[below] {$\scriptstyle \omega_N$};
        \draw (3.5,0.1) -- (3.5,0) node[below] {$\scriptstyle \omega_S$};
        \draw (3.5,0) edge[bend right=30,->] (2.5,0);
        \node at (3,0.3) {$\scriptstyle 1/2$};

        \node[right] at (3,2) {\footnotesize Continuous time};
        \node[right,red] at (3,1.7) {\footnotesize Discrete time};
    \end{tikzpicture}
\end{figure}

If $\Delta T$ is smaller, $\omega_S$ larger:

\begin{figure}[H]
    \centering
    \begin{tikzpicture}[node distance=2cm,auto,>=latex',declare function={
        a(\x) = (\x-0.5)^2+2;
        b(\x) = -(\x-1)^2+2.125;
        c(\x) = 2*(\x-2.5)^2+0.625;
        d(\x) = -(\x-3)^2+0.794;
        f(\x) = e^(-\x+2.9517);
        z(\x) = (\x < 0.5) * 2 +
                (0.5 <= \x) * (\x < 0.75) * a(\x) +
                (0.75 <= \x) * (\x < 2) * b(\x) +
                (2 <= \x) * (\x < 2.64) * c(\x) +
                (2.64 <= \x) * (\x < 3.35) * d(\x) +
                (3.35 <= \x) * f(\x);
        z2(\x) = z(\x) * (\x < 2.5) + z(\x) * (\x >= 2.5) * (\x < 4) * (1-(\x-2.5)^2/1.5/4);
    }]
        \draw[->] (0,0) -- (5.5,0) node[below] {$\omega$};
        \draw[->] (0,0) -- (0,3) node[left] {$|G|$};
        \draw[samples=50,domain=0:5,variable=\x,smooth] plot ({\x},{z(\x)});
        \draw[samples=50,red,domain=0:4,variable=\x,smooth] plot ({\x},{z2(\x)});

        \draw[dotted] (4,0.35) -- (4,0) node[below] {$\scriptstyle \omega_N$};
        \draw (4.8,0.1) -- (4.8,0) node[below] {$\scriptstyle \omega_S$};
        \draw (4.8,0) edge[bend right=30,->] (4,0);

        \node[align=center] at (4,3) {\footnotesize The discrete time approximation\\\footnotesize is valid over a larger bandwidth};
    \end{tikzpicture}
\end{figure}

\[
    \Delta T \rightarrow f_S = \frac{1}{\Delta T} \qquad \omega_S = \frac{2\pi}{\Delta T} \qquad f_N = \frac{1}{2} f_S \qquad \omega_N = \frac{1}{2} \omega_S
\]

Hidden problems of a too-small $\Delta T$:
\begin{itemize}
    \item Sampling devices (A/D and D/A) cost
    \item Computational cost: update an algorithm every $1 \mu s$ is much heavier than every $1 ms$
    \item Cost of memory (if data logging is needed)
    \item Numerical precision \emph{cost} (hidden computational cost)
\end{itemize}

\begin{figure}[H]
    \centering
    \begin{tikzpicture}[node distance=2cm,auto,>=latex']
        \draw[->] (-2,0) -- (2,0) node[above] {$\operatorname{Re}$};
        \draw[->] (0,-2) -- (0,2) node[left] {$\operatorname{Im}$};

        \draw[->] (4,0) -- (8,0) node[above] {$\operatorname{Re}$};
        \draw[->] (6,-2) -- (6,2) node[left] {$\operatorname{Im}$};

        \draw (6,0) circle (1cm);

        \node[cross] at (-0.5,0) {};
        \node[cross] at (-1,0) {};

        \draw (-0.5,0) edge[dotted, bend left=30,->] (6.99,0);
        \draw (-1,0) edge[dotted, bend left=30,->] (6.98,0);
        \node[right] at (7,-0.2) {$\scriptstyle 0.99995$};
        \node[right] at (7,-0.4) {$\scriptstyle 0.99996$};

        \node at (0,2.5) {$s$-domain};
        \node at (6,2.5) {$z$-domain};

        \draw[->] (2,2) -- (4,2) node[pos=0.5] {$\lambda_F=e^{\lambda_A\Delta T}$};
    \end{tikzpicture}
\end{figure}

If $\Delta T$ is very small (tends to zero), we squeeze all the poles very closed to $(1,0)$. We need very high numerical precision (use a lot of digits) to avoid instability.

Rule of thumb of control engineers: $f_S$ is between 10 and 20 times the system bandwidth we are interested in.

\begin{figure}[H]
    \centering
    \begin{tikzpicture}[node distance=2cm,auto,>=latex',declare function={
        a(\x) = (\x-0.5)^2+2;
        b(\x) = -(\x-1)^2+2.125;
        c(\x) = 2*(\x-2.5)^2+0.625;
        d(\x) = -(\x-3)^2+0.794;
        f(\x) = e^(-\x+2.9517);
        z(\x) = (\x < 0.5) * 2 +
                (0.5 <= \x) * (\x < 0.75) * a(\x) +
                (0.75 <= \x) * (\x < 2) * b(\x) +
                (2 <= \x) * (\x < 2.64) * c(\x) +
                (2.64 <= \x) * (\x < 3.35) * d(\x) +
                (3.35 <= \x) * f(\x);
    }]
        \draw[->] (0,0) -- (5.5,0) node[below] {$\omega$};
        \draw[->] (0,0) -- (0,3) node[left] {$|G|$};
        \draw[samples=50,domain=0:5,variable=\x,smooth] plot ({\x},{z(\x)});

        \draw[dotted] (1.5,1.875) -- (1.5,0) node[below] {};
        \node[below] at (0.75,0) {\footnotesize B.W. of interest};
        \draw (4,0.1) -- (4,0) node[below] {$\scriptstyle f_S$};
        \draw (1.5,0) edge[bend left=20,->] (4,0);
        \node at (2.75,0.45) {\footnotesize x10};
    \end{tikzpicture}
\end{figure}

\begin{remark}[Another way of managing the choice of $\Delta T$ w.r.t. the aliasing problem]
    \begin{figure}[H]
        \centering
        \begin{tikzpicture}[node distance=2cm,auto,>=latex',declare function={
            f(\x) = 2.7 * (1 - \x/4) + rand^2/4;
        }]
            \draw[->] (0,0) -- (5.5,0) node[below] {$\omega$};
            \draw[->] (0,0) -- (0,3) node[left] {$\Gamma_{\tilde{x}}(\omega)$};
            \draw[domain=0:4,samples=40,variable=\x,smooth] plot ({\x},{f(\x)});
            \node[below, align=center] at (2,0) {BW of the full spectral\\content of $\tilde{x}(t)$};
            \draw (4,0.05) -- (4,-0.05);
        \end{tikzpicture}
    \end{figure}

    The classical way to deal with aliasing is to use anti-alias analog filters

    \begin{figure}[H]
        \centering
        \begin{tikzpicture}[node distance=2cm,auto,>=latex']
            \node[int,align=center] at (0,0) (sys) {Analog\\anti-alias\\low-pass\\filter};
            \node[int] at (2.5,0) (ADC) {A/D};

            \draw[->] (-2,0) -- (sys) node[pos=0.5] {$\tilde{x}(t)$};
            \draw[->] (sys) -- (ADC);
            \draw[->] (ADC) -- ++(1.5,0);
        \end{tikzpicture}
    \end{figure}

    For example, if A/D is at $1KHz$ ($\Delta T = 1ms$)
    \begin{figure}[H]
        \centering
        \begin{tikzpicture}[node distance=2cm,auto,>=latex',declare function={
            f(\x) = 2.7 * (1 - \x/3) + rand^2/3;
        }]
            \pgfmathsetseed{43}
            \draw[->] (0,0) -- (4,0) node[below] {$\omega$};
            \draw[->] (0,0) -- (0,3);
            \draw[domain=0:3,samples=40,variable=\x,smooth] plot ({\x},{f(\x)});
            \draw (3,0.05) -- (3,-0.05) node[below] {$\scriptstyle 2KHz$};

            \pgfmathsetseed{43}
            \draw[->] (5,0) -- (9,0) node[below] {$\omega$};
            \draw[->] (5,0) -- (5,3);
            \draw[domain=0:1,samples=13,variable=\x,smooth] plot ({\x+5},{f(\x)});
            \draw (6,0.05) -- (6,-0.05) node[below] {$\scriptstyle 0.5KHz$};
            \draw[dotted] (6,0) -- (6,2.1);

            \node[align=center] at (9,2) {Nyquist generates no\\alias at 1KHz};
        \end{tikzpicture}
    \end{figure}

    \paragraph{Full Digital Approach} without analog anti-alias filter

    \begin{figure}[H]
        \centering
        \begin{tikzpicture}[node distance=2cm,auto,>=latex',declare function={
            f(\x) = 2.7 * (1 - \x/3) + rand^2/3;
        }]
            \node[int] at (0,5) (ADC) {A/D};
            \node[int,align=center] at (3,5) (filter) {Digital low\\pass filter};
            \node[int,align=center] at (6,5) (sampl) {Under\\sampling};

            \draw[->] (-1.5,5) -- (ADC) node[pos=0.5] {$\tilde{x}(t)$};
            \draw[->] (ADC) -- (filter);
            \draw[->] (filter) -- (sampl);
            \draw[->] (sampl) -- ++(2,0);

            \node at (0,6) {$f_S=4KHz$};
            \node at (3,6) {Cutoff $0.5KHz$};
            \node at (6,6) {$f_S=1KHz$};

            \draw [decorate,decoration={brace,amplitude=10pt},yshift=-0.1cm] (7,4.5) -- (-0.5,4.5) node [black,midway,yshift=-0.4cm] {\footnotesize fully digital};

            \pgfmathsetseed{43}
            \draw[->] (0,0) -- (7,0) node[below] {};
            \draw[->] (0,0) -- (0,3);
            \draw[domain=0:3,samples=40,variable=\x,smooth] plot ({\x},{f(\x)});
            \draw (3,0.05) -- (3,-0.05) node[below] {$\scriptstyle 2KHz$};
            \draw (6.5,0.05) -- (6.5,-0.05) node[below] {$f_S$};
        \end{tikzpicture}
    \end{figure}
\end{remark}
