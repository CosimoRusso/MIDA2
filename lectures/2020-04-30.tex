\newlecture{Sergio Savaresi}{30/04/2020}

\begin{remark}[Emphasis on special frequency range]
    In some cases, between $\omega_1$ and $\omega_H$, we want to be more accurate in system identification in some frequency regions (typically around cut-off-frequency, around resonances).

    We can use different weights for different frequencies.

    \begin{figure}[H]
        \centering
        \begin{tikzpicture}[node distance=2.5cm,auto,>=latex']
            \begin{axis}[axis lines=none,ymax=2.5]
                \addplot[color=blue,mark=square]
                    coordinates {
                        (0,1)(1,1)(2,1)(3,1.2)(4,2)(5,1.2)(6,1)(7,1)(8,1)
                    };
            \end{axis}

            \draw[->] (-0.5,0) -- (7,0) node[right] {$\omega$};
            \draw[->] (0.5,-0.5) -- (0.5,5) node[left] {$\lambda_i$};
            \node at (0.1,0.5) {$1$};
            \draw (6.3,0.1) -- (6.3,-0.1) node [below] {$\omega_H$};
            \draw (1.3,0.1) -- (1.3,-0.1) node [below] {$\omega_1$};
            \draw (2,0.1) -- (2,-0.1) node [below] {$\omega_2$};
        \end{tikzpicture}
    \end{figure}

    The performance index can be redefined:
    \[
        \tilde{J}_H (\theta) = \frac{1}{H} \sum_{i=1}^H \gamma_i \left(W(e^{j\omega_i};\theta) - \frac{\hat{B}_i}{A_i}e^{j\hat{\phi_i}}\right)^2
    \]

    Another \emph{trick}: more dense $\omega_i$ spacing in the frequency region of special interest (not really used).
\end{remark}

\begin{remark}[Single experiment]
    Sometimes the set of $H$ independent single-sinusoid experiments can be replaced by a long single ``sine-sweep'' experiment.

    \begin{figure}[H]
        \centering
        \begin{tikzpicture}[node distance=2.5cm,auto,>=latex']
            \draw[->] (-0.5,0) -- (5,0) node[right] {$t$};
            \draw[->] (0,-1) -- (0,2) node[left] {$u(t)$};
            \draw[domain=0:4.5,samples=100,smooth,variable=\x,red] plot ({\x},{cos(\x*180/3.14*(\x^1.7+5))*e^(-\x/2.5)});
        \end{tikzpicture}
        \caption*{Slowing-varying sinusoid with increasing frequency and decreasing amplitude.}
    \end{figure}

    We can cut a-posteriori the signal into $H$ pieces, and then back to the standard procedure.
    Or directly compute an estimation of $\hat{W}(e^{j\omega})$ as a ration of the output and input spectra.

    \[
        \hat{W}(e^{j\omega}) = \frac{\hat{\Gamma}_y(e^{j\omega})}{\hat{\Gamma}_u(e^{j\omega})}
    \]

    We can fit the estimated $\hat{W}(e^{j\omega})$ with the model frequency response $W(e^{j\omega}, \theta)$ in the performance index.
    This experiment is quicker but has usually a lower signal-to-noise-ration.
\end{remark}

\section{Comparison between time domain (ARMAX) and frequency domain parametric methods}

\paragraph{Frequency domains}
\begin{description}
    \item[Pro] Robust and very reliable. We put a lot of energy on each sinusoid. The frequency response estimation is very reliable.
    \item[Pro] Intuitive (easy to understand)
    \item[Pro] Consistent with control-design methods (usually in frequency domain)
    \item[Cons] More demanding for the experiment
\end{description}

Note that F.D. and T.D. methods should provide the same result if done correctly.


\chapter{Kalman Filter (software sensing in feedback)}

In MIDA1 we have mostly used I/O transfer function representations:
\[ y(t) = \frac{B(z)}{A(z)}u(t-k) + \frac{C(z)}{A(z)}e(t) \qquad e(t) \sim WN \]


Kalman filter theory is fully based on S.S. representation.

\[
    \begin{cases}
        x(t+1) = Fx(t) + Gu(t) + v_1(t)  & \qquad v_1 \sim WN \\
        y(t) = Hx(t) + \cancel{Du(t)} + v_2(t) & \qquad v_2 \sim WN
    \end{cases}
\]

The noise $v_1(t)$ represents how much we do not know about the state, while $v_2(t)$ represents the noise in the measurement: how "bad" the sensors are.

\section{Motivation and Goals}

Given a model description $\{F, G, H\}$ and noises variances (not a system identification technique), with K.F. theory we can address the following problems:

\begin{itemize}
    \item $k$-steps ahead prediction of output: $\hat{y}(t+k|t)$ (already solved in MIDA1 for ARMAX).
    \item $k$-steps ahead prediction of state: $\hat{x}(t+k|t)$ (at time $t$ we have available $y(t)$, $y(t-1)$, $\ldots$, $u(t)$, $u(t-1)$, $\ldots$).
    \item Find the filter of the state: $\hat{x}(t|t)$ (given data of $y(t)$ and past, and $u(t)$ and part, the estimation is made at the same time). In practice it's \emph{software-sensing}, most important problem solved by Kalman filter (ideed K.F. is named \emph{filter}).
    \item Gray box system identification (see chapter 5). We have a recorded data set and the model structure with some unknown parameters.
\end{itemize}


Dynamical systems have this layout:
\begin{figure}[H]
    \centering
    \begin{tikzpicture}[node distance=2.5cm,auto,>=latex']
        \node [int, ellipse, align=center] (sys) {$x_1(t)$\\$x_2(t)$\\$x_n(t)$};
        \node (in1) [left of=sys, node distance=2cm, yshift=0.7cm] {$u_1(t)$};
        \node (in2) [left of=sys, node distance=2cm] {$u_2(t)$};
        \node (in3) [left of=sys, node distance=2cm, yshift=-0.7cm] {$u_m(t)$};
        \node (out1) [right of=sys, node distance=2cm, yshift=0.7cm]{$y_1(t)$};
        \node (out2) [right of=sys, node distance=2cm]{$y_2(t)$};
        \node (out3) [right of=sys, node distance=2cm, yshift=-0.7cm]{$y_p(t)$};

        \draw[->] (in1) -- (sys);
        \draw[->] (in2) -- (sys);
        \draw[->] (in3) -- (sys);
        \draw[->] (sys) -- (out1);
        \draw[->] (sys) -- (out2);
        \draw[->] (sys) -- (out3);
    \end{tikzpicture}
\end{figure}

MIMO system with $m$ inputs, $p$ outputs and $n$ states.

\textbf{Key problem} usually $p \ll n$, physical sensors are much less than system states.
\begin{itemize}
    \item Cost
    \item Cables, power supply
    \item Maintenance (faults, degradation)
\end{itemize}

It is useful to have full ``measurement'' of states because:
\begin{itemize}
    \item Control design (state feedback design)
    \item Monitoring (fault detection, predictive maintenance, \dots)
\end{itemize}

Software sensing (also called virtual sensing):
\begin{figure}[H]
    \centering
    \begin{tikzpicture}[node distance=2.5cm,auto,>=latex']
        \node [int, ellipse, align=center] (sys) {system\\$x(t)$};
        \node (in1) [left of=sys, node distance=2cm, yshift=0.7cm] {};
        \node (in2) [left of=sys, node distance=2.3cm] {$u(t)$};
        \node (in3) [left of=sys, node distance=2cm, yshift=-0.7cm] {};
        \node (out1) [right of=sys, node distance=2cm, yshift=0.7cm]{};
        \node (out2) [right of=sys, node distance=2.3cm]{$y(t)$};
        \node (out3) [right of=sys, node distance=2cm, yshift=-0.7cm]{};

        \node [int, ellipse, align=center, below of=sys] (algo) {algorithm in\\a DCD (KF)};
        \node (algoout) [right of=algo, node distance=3cm] {$\hat{x}(t)$};

        \draw[->] (in1) -- (sys);
        \draw[->] (in2) -- (sys);
        \draw[->] (in3) -- (sys);
        \draw[->] (sys) -- (out1);
        \draw[->] (sys) -- (out2);
        \draw[->] (sys) -- (out3);

        \draw[->] (-1.4,0.8) -- (-1.4,-1) -- (algo);
        \draw[->] (1.4,0.8) -- (1.4,-1) -- (algo);
        \draw[->] (algo) -- (algoout);
    \end{tikzpicture}
\end{figure}

\paragraph{Dilemma} When using software sensing and when physical sensing?
\begin{itemize}
    \item In some cases there is no option (not feasible installation of a physical sensor)
    \item In most cases both options are viable: variable vs fixed cost
\end{itemize}

\begin{figure}[H]
    \centering
    \begin{tikzpicture}[node distance=2.5cm,auto,>=latex']
        \draw[->] (-0.5,0) -- (4,0) node[right, align=center] {Fixed development costs\\of software sensing algorithm};
        \draw[->] (0,-0.5) -- (0,4) node[left, align=right] {Variable cost proportional\\to the number of sensors};
        \draw[domain=0:2.5,samples=20,smooth,variable=\x,red] plot ({\x},{sqrt(2.5^2-\x^2)});
        \draw [decorate,decoration={brace,amplitude=10pt}] (0,0) -- (0,2.5) node [black,midway,align=right,xshift=-0.3cm] {Low volumes\\better physical sensing};
        \draw [decorate,decoration={brace,amplitude=7pt}] (0,3.5) -- (0,2.5) node [black,midway,align=left,xshift=0.3cm] {High volumes\\better software sensing};
    \end{tikzpicture}
    \caption*{Break-even analysis}
\end{figure}
